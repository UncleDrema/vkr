\chapter*{ВВЕДЕНИЕ}
\addcontentsline{toc}{chapter}{ВВЕДЕНИЕ}
\iffalse
Современные многоагентные системы (МАС) находят широкое применение в различных областях: от робототехнических платформ~\cite{ieee-mas} до сложных симуляций в видеоиграх~\cite{rts-potential-fields} и систем управления распределёнными ресурсами~\cite{robustAndScalableCoordinatonOfPotentialFields}.
Координация агентов в таких системах является одной из ключевых задач~\cite{coordination-book}, поскольку от её успешного решения зависит эффективность работы всей системы. 

Координация агентов требует учёта множества факторов, включая динамическую природу среды, ограниченность вычислительных ресурсов и необходимость работы в реальном времени~\cite{stone2000multiagent}.
На практике это приводит к необходимости разработки методов, способных обеспечивать слаженность действий агентов даже в условиях неопределённости и ограниченного взаимодействия между ними~\cite{wooldridge2009introduction}.

Целью данной научно-исследовательской работы является сравнение методов координации агентов применительно к задаче визуального контроля критических областей.

Для достижения поставленной цели необходимо решить следующие задачи:
\begin{enumerate}
	\item Провести анализ предметной области и описать рассматриваемую многоагентную систему.
	\item Выделить характеристики для классификации и сравнения методов координации агентов в многоагентных системах.
	\item Формализовать математические описания рассматриваемых методов.
	\item Провести сравнительный анализ методов по ключевым характеристикам.
\end{enumerate}

Результаты исследования послужат основой для разработки эффективного метода координации агентов в задаче визуального контроля критических областей.
\fi