\chapter{Исследовательская часть}
В данном разделе поставлена цель исследования, описано проведенное исследование и сделанные на его основе выводы.

\section{Цель исследования}
Цель исследования -- удостовериться в корректности генерируемых в ходе преобразования SQL-выражений исходных кодов и соответствия выполняемых ими операций семантике исходного SQL-выражения.

Типы запросов \texttt{CREATE}:
\begin{itemize}
	\item Создание таблицы с простым первичным ключом и размером строки до 64 бит;
	\item Создание таблицы с простым первичным ключом и размером строки более 64 бит;
	\item Создание таблицы с составным первичным ключом и размером строки до 64 бит;
	\item Создание таблицы с составным первичным ключом и размером строки более 64 бит.
\end{itemize}

Типы запросов \texttt{SELECT}:
\begin{itemize}
	\item Выборка столбца из таблицы;
	\item Выборка столбца из таблицы с условием, включащим другой столбец;
	\item Выборка столбца из таблицы с условием, включащим параметр;
	\item Выборка столбцов из таблицы с условием, включащим другой столбец;
	\item Выборка столбцов из таблицы с условием, включащим параметр;
	\item Выборка из таблицы с составным условием, включающим несколько столбцов;
	\item Выборка из таблицы с составным условием, включающим параметр;
	\item Выборка из таблицы с ограничением числа результатов.
\end{itemize}

Типы запросов \texttt{INSERT}:
\begin{itemize}
	\item Вставка одной строки с константными значениями;
	\item Вставка одной строки с параметрмаи;
	\item Вставка нескольких строк.
\end{itemize}

Проверка корректности сгенерированных исходных кодов проводится путем их компиляции и загрузки в качестве программного ядра \texttt{lnh64} и запуска обработчиков при помощи запроса из host-подсистемы.

\section{Результаты исследования}
В данном разделе представлены результаты исследования корректности генерируемых инструкций для трех поддерживаемых типов SQL-выражений.

\subsection{Создание таблицы}
Корректность генерации структур хранения данных проверим путем запуска кодогенерации и сравнения результата с ожидаемым -- таблица \ref{tbl:research_create}.

\begin{table}[H]
	\caption{Результат исследования создания таблицы}
	\label{tbl:research_create}
	\begin{tabular}{|p{6.5cm}|p{3.3cm}|p{3.3cm}|}
		\hline
		Запрос & Ожидаемый результат & Результат \\ \hline
		Создание таблицы с простым первичным ключом и размером строки 64 бита & Структура из одной группы & Структура из одной группы \\ \hline
		Создание таблицы с простым первичным ключом и размером строки 128 бит & Структура из двух групп & Структура из двух групп \\ \hline
		Создание таблицы с составным первичным ключом и размером строки 64 бита & Структура из одной группы & Структура из одной группы \\ \hline
		Создание таблицы с составным первичным ключом и размером строки 128 бит & Структура из двух групп & Структура из двух групп \\ \hline
	\end{tabular}
\end{table}

\subsection{Выборка данных}
Для тестирования сгенерированных итераторов языка \texttt{C++} на сооветствие запросу база данных была создана структура хранения данных при помощи кодогенерации на основе SQL-выражения, представленного в листинге \ref{lst:users.sql}. данными, указанными в таблице \ref{tbl:data}.

\includelistingpretty
{users.sql}
{SQL}
{SQL-выражение создания таблицы пользователей}

Заполнение базы данных было произведено тестовыми данными, приведенными в таблице \ref{tbl:data}.

\begin{table}[H]
	\caption{Тестовые данные}
	\label{tbl:data}
	\begin{tabular}{|c|c|c|c|}
		\hline
		Id & User & Role & Time \\ \hline
		0 & 0 & 0 & 0 \\ \hline
		1 & 1 & 1 & 1000 \\ \hline
		2 & 1 & 1 & 2000 \\ \hline
		3 & 1 & 1 & 3000 \\ \hline
		4 & 1 & 1 & 4000 \\ \hline
		5 & 2 & 4 & 999 \\ \hline
		6 & 6 & 5 & 6000 \\ \hline
		7 & 7 & 6 & 7000 \\ \hline
		8 & 8 & 7 & 8000 \\ \hline
		9 & 9 & 8 & 9000 \\ \hline
	\end{tabular}
\end{table}

Далее в листингах \ref{lst:select1.sql}-\ref{lst:select9.sql} приведены SQL-выражения и результаты их выполнения в виде числовых значений запрошенных столбцов для каждой строки, попавшей в выборку.
\pagebreak

\includelistingpretty
{select1.sql}
{SQL}
{Выборка столбца из таблицы}

\includelistingpretty
{select2.sql}
{SQL}
{Выборка столбца из таблицы с условием с другим столбцом}

\includelistingpretty
{select3.sql}
{SQL}
{Выборка столбца из таблицы с условием с параметром}
\pagebreak
\includelistingpretty
{select4.sql}
{SQL}
{Выборка столбцов из таблицы}

\includelistingpretty
{select5.sql}
{SQL}
{Выборка столбцов из таблицы с условием с другим столбцом}

\includelistingpretty
{select6.sql}
{SQL}
{Выборка столбцов из таблицы с условием с параметром}
\pagebreak
\includelistingpretty
{select7.sql}
{SQL}
{Выборка из таблицы с составным условием с разными столбцами}

\includelistingpretty
{select8.sql}
{SQL}
{Выборка из таблицы с составным условием с параметром}

\includelistingpretty
{select9.sql}
{SQL}
{Выборка из таблицы с ограничением числа результатов}

Все результаты выполнения запросов соответствуют ожидаемым, следовательно, данные типы запросов корректно обрабатываются кодогенератором.

\subsection{Вставка данных}
Вставка данных была протестирована при выполнении тестирования выборки данных, поскольку заполнение базы данных происходило путём вызова выражений вставки трёх видов.
Соответствующие SQL-выражения приведены в листинге \ref{lst:inserts.sql}.

\includelistingpretty
{inserts.sql}
{SQL}
{SQL-выражения вставки}

\section*{Вывод по исследовательской части}
В результате исследования показано, что разработанная реализация метода кодогенерации успешно создает инструкции доступа к данным в вычислительном комплексе <<Тераграф>> из SQL-выражений и позволяет осуществлять работу с ним при помощи базовых инструкций создания таблиц, вставки и выборки данных.