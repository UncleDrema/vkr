\chapter{Сравнение методов координации агентов в многоагентных системах}

На основе классификации методов, описанных выше, проведём сравнительный анализ по критериям, сформулированным ранее. Таблица \ref{tab:comparison} суммирует ключевые свойства методов координации.

\begin{landscape}
	\begin{table}[H]
		\centering
		\caption{Сравнительный анализ методов координации агентов в многоагентной системе}
		\label{tab:comparison}
		\begin{tabular}{|c|c|c|c|c|c|c|}
			\hline
			\multicolumn{4}{|c|}{\textbf{Часть 1: Тип взаимодействия, Область восприятия, Распределение задач}} \\ \hline
			\textbf{Метод} & \textbf{Тип взаимодействия} & \textbf{Область восприятия} & \textbf{Распределение задач} \\ \hline
			Потенциалов    & Децентрализованный         & Локальная                  & Динамическое                 \\ \hline
			Ролей          & Гибридный                  & Глобальная                 & Динамическое                 \\ \hline
			Роя            & Децентрализованный         & Локальная                  & Динамическое                 \\ \hline
			Теории игр     & Гибридный                  & Глобальная                 & Статическое                  \\ \hline
			Обучения       & Гибридный                  & Глобальная                 & Статическое                  \\ \hline
			\multicolumn{4}{|c|}{} \\ % Пустая строка для визуального разделения
			\hline
			\multicolumn{4}{|c|}{\textbf{Часть 2: Сложность, Гибкость, Правдоподобность}} \\ \hline
			\textbf{Метод} & \textbf{Сложность}         & \textbf{Гибкость}          & \textbf{Правдоподобность}    \\ \hline
			Потенциалов    & $O(a \cdot (c + b + t))$   & Высокая                    & Средняя                      \\ \hline
			Ролей          & $O(a \cdot (v + c + t))$   & Высокая                    & Высокая                      \\ \hline
			Роя            & $O(a \cdot (|N_i| + v + t))$ & Высокая                    & Высокая                      \\ \hline
			Теории игр     & $O(k \cdot a \cdot (v + e + |S_i|))$ & Средняя          & Высокая                      \\ \hline
			Обучения       & $O_{\text{обучение}}$      & Высокая                    & Высокая                      \\ \hline
		\end{tabular}
	\end{table}
\end{landscape}


\subsubsection*{Анализ результатов}
Рассмотрим свойства методов с учётом требований задачи:

1. **Потенциальные поля**:  
Метод обладает высокой гибкостью и сравнительно низкой сложностью, что делает его подходящим для задач в реальном времени. Однако правдоподобность поведения ограничена, так как агенты могут демонстрировать «нереалистичные» траектории движения из-за наличия локальных минимумов.

2. **Метод ролей**:  
Метод обеспечивает высокую правдоподобность благодаря детальному распределению задач, но страдает от низкой гибкости, так как агенты строго следуют заданным ролям. Также сложность возрастает с числом препятствий.

3. **Рой**:  
Метод роя демонстрирует высокую гибкость и устойчивость к изменениям среды, что полезно для динамических задач. Однако его квадратичная сложность по числу агентов ограничивает масштабируемость для больших систем.

4. **Планирование на основе теории игр**:  
Метод предоставляет наиболее правдоподобное поведение агентов благодаря равновесным стратегиям. Однако сложность метода остаётся высокой, особенно для больших систем с большим числом стратегий.

\subsubsection*{Рекомендации}
Для решения задачи координации в режиме реального времени рекомендуется использовать метод потенциальных полей или рой, учитывая их низкую сложность и высокую гибкость. Для задач, требующих высокой правдоподобности поведения (например, симуляции сложных игровых сценариев), более подходящим будет метод ролей или планирование на основе теории игр.