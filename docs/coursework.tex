\documentclass{bmstu}
\bibliography{biblio}
\usepackage{graphicx}
\usepackage{subcaption}
\usepackage{amsmath}
\usepackage{adjustbox}
\usepackage{rotating}
\usepackage{makecell}
\usepackage{pdflscape}
\newcommand{\norm}[1]{\lvert #1 \rvert}
\renewcommand{\labelitemi}{---}
\renewcommand{\labelitemii}{$\bullet$}
\renewcommand{\labelitemiii}{$\circ$}

\usepackage{multirow}
\usepackage{tocloft}
\usepackage{booktabs}

\renewcommand{\includelistingpretty}[3]{
	\lstinputlisting[caption={#3}, label={lst:#1}, language={#2}]{./inc/lst/#1}
}

\lstset{
	basicstyle=\ttfamily\footnotesize,
	frame=single,
	tabsize=2,
	linewidth=0.95\textwidth,
	xleftmargin=0.025\textwidth,
	numbers=left,
	numbersep=5pt,
	breaklines=true,
	keepspaces=true,
	showspaces=false,
	showstringspaces=false,
	breakatwhitespace=false,
	float=h!,
	abovecaptionskip=-5pt,
}

\begin{document}

\makethesistitle
{Информатика и системы управления}
{Программное обеспечение ЭВМ и информационные технологии}
{Метод координации агентов в игровой системе безопасности с использованием потенциальных полей}
{Дремин~К.~А./ИУ7-86Б}
{Москвичев~Н.~В.} % ФИО научного руководителя
{}
{Кострицкий~А.~С.}

\newenvironment{essay2}[1]
{
	\chapter*{РЕФЕРАТ}
	
	Расчетно-пояснительная записка \begin{NoHyper}30\end{NoHyper}~с., \tottab~табл., \thetotbibentries~источн.,  1~прил.
	
	\MakeUppercase{#1} \par
}{}

\setcounter{page}{5}
\begin{essay2}{загуражемый модуль, промахи кеша, TLB.}
	Работа посвящена разработке загружаемого модуля ядра Linux для сбора статистики промахов в кеше TLB.
	
	\iffalse
	В результате работы был проведен анализ предметной области координации агентов в многоагентных системах, описаны основные методы решения задачи.
	Сформулированы критерии сравнения и классификации описанных методов и выполнен их сравнительный анализ.
	\fi
\end{essay2}

%\maketableofcontents

\clearpage
\cftsetindents{chapter}{\parindent}{12.5mm}
\cftsetindents{section}{\parindent}{12.5mm}
\cftsetindents{subsection}{\parindent}{12.5mm}
\setlength{\cftbeforetoctitleskip}{-22pt}
\renewcommand{\cftaftertoctitle}{\hfill}
\renewcommand{\contentsname}{\hfill\LARGE СОДЕРЖАНИЕ\hfill}
\tableofcontents

%\begin{definitions}
	\definition{Трассировка сфер}{техника отрисовки в трёхмерной компьютерной графике, основанная на итеративном обходе лучей света с шагом, равным расстоянию от текущей точки на луче до ближайшего объекта}
\end{definitions}
%\begin{abbreviations}
	\definition{ФРЗ}{Функция расстояния со знаком}
	\definition{SDF}{Signed distance function, то же самое, что и ФРЗ}
\end{abbreviations}

\chapter*{ВВЕДЕНИЕ}
\addcontentsline{toc}{chapter}{ВВЕДЕНИЕ}
\iffalse
Современные многоагентные системы (МАС) находят широкое применение в различных областях: от робототехнических платформ~\cite{ieee-mas} до сложных симуляций в видеоиграх~\cite{rts-potential-fields} и систем управления распределёнными ресурсами~\cite{robustAndScalableCoordinatonOfPotentialFields}.
Координация агентов в таких системах является одной из ключевых задач~\cite{coordination-book}, поскольку от её успешного решения зависит эффективность работы всей системы. 

Координация агентов требует учёта множества факторов, включая динамическую природу среды, ограниченность вычислительных ресурсов и необходимость работы в реальном времени~\cite{stone2000multiagent}.
На практике это приводит к необходимости разработки методов, способных обеспечивать слаженность действий агентов даже в условиях неопределённости и ограниченного взаимодействия между ними~\cite{wooldridge2009introduction}.

Целью данной научно-исследовательской работы является сравнение методов координации агентов применительно к задаче визуального контроля критических областей.

Для достижения поставленной цели необходимо решить следующие задачи:
\begin{enumerate}
	\item Провести анализ предметной области и описать рассматриваемую многоагентную систему.
	\item Выделить характеристики для классификации и сравнения методов координации агентов в многоагентных системах.
	\item Формализовать математические описания рассматриваемых методов.
	\item Провести сравнительный анализ методов по ключевым характеристикам.
\end{enumerate}

Результаты исследования послужат основой для разработки эффективного метода координации агентов в задаче визуального контроля критических областей.
\fi
\chapter{Аналитический раздел}

\section{Постановка задачи координации агентов в многоагентной системе}
%\addcontentsline{toc}{section}{Постановка задачи координации агентов в многоагентной системе}

Координация агентов в многоагентных системах (МАС) представляет собой задачу организации взаимодействия между автономными субъектами (агентами) с целью достижения общей цели или выполнения множества задач~\cite{wooldridge2009introduction}~\cite{coordination-book}.
В данной работе задача координации рассматривается в следующих условиях:
\begin{itemize}
	\item общей целью агентов является визуальный контроль областей среды и минимизация угроз нарушения их безопасности;
	\item рассматриваемая МАС является частью игрового приложения, вследствие чего необходимо достичь внешне реалистичного поведения агентов;
	\item метод координации агентов не должен быть слишком трудоемким -- теоретическая реализация выбранного алгоритма должна быть пригодна для применения в игровых приложениях.
\end{itemize}

Рассматриваемая в рамках данной работы многоагентная система относится к следующим категориям, описанным~в~\cite{ieee-mas}:
\begin{enumerate}
	\item \textbf{Лидерство:}  
	Система является \textbf{безлидерной}, так как агенты действуют независимо, принимая решения на основе локальных данных и целей, согласованных с общей моделью.
	
	\item \textbf{Функция принятия решений:}  
	Принятие решений в данной МАС \textbf{нелинейное}, так как действия агентов зависят от сложных взаимодействий между областями, угрозами и агентами.
	
	\item \textbf{Гетерогенность:}  
	Система является \textbf{гомогенной}, так как все агенты обладают одинаковыми возможностями и характеристиками.
	
	\item \textbf{Топология:}  
	Топология системы \textbf{динамическая}, так как агенты перемещаются в пространстве и их взаимодействия изменяются в зависимости от положения и состояния среды.
	
	\item \textbf{Мобильность:}  
	Агенты в системе \textbf{мобильные}, так как перемещаются в пространстве для выполнения своих задач, таких как патрулирование и реагирование на угрозы.
\end{enumerate}

\section{Описание среды координации}
Рассмотрим формальное описание среды, в которой функционируют агенты при решении данной задачи, чтобы в дальнейшем определить наиболее применимые методы.

\textbf{1. Доступные для перемещения области}.  
Среда моделируется как множество областей $\mathcal{N} \subset \mathbb{R}^2$, доступных для перемещения агентов, описываемых навигационной картой. Навигационная карта представлена графом $G = (V, E)$~\cite{overview-of-recent}~\cite{ieee-mas}, где $V$ — множество вершин, соответствующих дискретным точкам в $\mathcal{N}$, а $E$ — множество ребер, определяющих пути перемещения между вершинами, как показано в формуле~(\ref{eq:area}):

\begin{equation}
\mathcal{N} = \bigcup_{v \in V} \mathcal{A}_v, \quad \mathcal{A}_v \subset \mathbb{R}^2,
\label{eq:area}
\end{equation}
где $\mathcal{A}_v$ — выпуклый многоугольник, ассоциированный с вершиной $v$.

\textbf{2. Препятствия, ограничивающие обзор}.  
Препятствия в среде $\mathcal{E}$ задаются множеством $\mathcal{B} = \{B_1, B_2, \dots, B_k\}$, где каждый объект $B_i$ определяется ограничивающей областью в пространстве $\mathbb{R}^2$ и определен в соответствии с формулой (\ref{eq:obstacles}):
\begin{equation}
B_i = [x_{\min}, x_{\max}] \times [y_{\min}, y_{\max}], \quad i \in \{1, 2, \dots, k\}.
\label{eq:obstacles}
\end{equation}
Присутствие объектов $\mathcal{B}$ влияет на обзор агентов, ограничивая видимость в направлении, пересекающем препятствия.

\textbf{3. Критические области}.  
Критические области, требующие визуального контроля, заданы множеством точек $\mathcal{C} = \{c_1, c_2, \dots, c_m\} \subset \mathcal{N}$.
Каждая точка $c_j$ характеризуется радиусом влияния $r_j > 0$, определяющим зону контроля, как описано в формуле (\ref{eq:critical_areas}):
\begin{equation}
\mathcal{Z}(c_j) = \{p \in \mathbb{R}^2 \mid \|p - c_j\| \leq r_j\}, \quad j \in \{1, 2, \dots, m\}.
\label{eq:critical_areas}
\end{equation}
Задача агентов заключается в том, чтобы обеспечить покрытие всех зон $\mathcal{Z}(c_j)$ при учете ограничения видимости, задаваемого препятствиями.

\textbf{4. Видимость в среде}.  
Модель видимости агента $a_i$ определяется его текущим положением $p_i \in \mathcal{N}$ и углом обзора $\phi_i$.
Область видимости агента формируется как сектор окружности, определяемый формулой (\ref{eq:visibility}):
\begin{equation}
\mathcal{V}(p_i, \phi_i, r_{\text{max}}) = \{p \in \mathbb{R}^2 \mid \|p - p_i\| \leq r_{\text{max}}, \, \theta(\dot{p}_i(t), p) \leq \phi_i\},
\label{eq:visibility}
\end{equation}
где $r_{\text{max}}$ — максимальная дальность обзора, $\theta(\dot{p}_i(t), p)$ — угол между направлением агента и вектором к точке $p$.

Таким образом, среда представляет собой совокупность доступных областей $\mathcal{N}$, препятствий $\mathcal{B}$ и критических точек $\mathcal{C}$, которые агенты обязаны контролировать с учетом ограничений видимости.
Основная задача координации заключается в определении таких траекторий и позиций агентов, которые минимизируют неохваченные зоны $\mathcal{Z}(c_j)$.

\textbf{5. Агенты}.  
Агенты в системе заданы множеством $\mathcal{A} = \{a_1, a_2, \dots, a_n\}$.
Каждый агент $a_i$ характеризуется следующими параметрами:  
\begin{itemize}[leftmargin=1.6\parindent]
	\item текущей позицией $p_i(t) \in \mathcal{N}$ в момент времени $t$;  
	\item направлением движения $\theta_i(t) \in [0, 2\pi)$;  
	\item скоростью перемещения $v_i(t) \in [0, v_{\text{max}}]$.
\end{itemize}

Траектория движения агента задается уравнением (\ref{eq:agent_movement}):  
\begin{equation}
\dot{p}_i(t) = v_i(t) \cdot \begin{bmatrix}
	\cos(\theta_i(t)) \\
	\sin(\theta_i(t))
\end{bmatrix}.
\label{eq:agent_movement}
\end{equation}

\textbf{6. Угрозы}.  
Угрозы представлены множеством $\mathcal{T}_{\text{threat}} = \{\tau_1, \tau_2, \dots, \tau_k\}$, где каждая угроза $\tau_j$ является динамическим объектом, имеющим параметры:  
\begin{itemize}[leftmargin=1.6\parindent]
	\item позицию $q_j(t) \in \mathcal{N}$ в момент времени $t$;  
	\item направление движения $\phi_j(t) \in [0, 2\pi)$;  
	\item скорость $v_j(t) \in [0, v_{\text{threat}}]$.;
	\item радиус влияния угрозы $r_{\tau}$.
\end{itemize}

Движение угроз также описывается уравнением (\ref{eq:agent_movement}).

Достижение угрозой одной из критических точек $c \in \mathcal{C}$ считается нарушением безопасности и должно быть предотвращено агентами.

\textbf{7. Динамическое патрулирование}.  
Для обеспечения контроля над всей областью $\mathcal{N}$ агенты реализуют стратегию динамического патрулирования.
При этом вводится функция \textit{опасности} $U(p, t)$, описывающая степень опасности в точке $p \in \mathcal{N}$ в момент времени $t$. 

Общая динамика изменения $U(p, t)$ описывается уравнением (\ref{eq:danger_function_patrol}):
\begin{equation}
	\frac{\partial U(p, t)}{\partial t} = \alpha - \beta \sum_{i=1}^n \mathbf{1}\{p \in \mathcal{V}(p_i(t), \phi_i(t), r_{\text{max}})\},
	\label{eq:danger_function_patrol}
\end{equation}
где:
$\alpha > 0$ — скорость естественного роста опасности в непосещаемых зонах;
$\beta > 0$ — скорость снижения опасности за счет патрулирования агентами.

В области с радиусом $r_\tau$ вокруг угрозы значение функции опасности увеличивается согласно формуле (\ref{eq:danger_function_threat}):
\begin{equation}
	U(p, t) = U(p, t) + \gamma \cdot \mathbf{1}\{\|p - q_\tau(t)\| \leq r_\tau\},
	\label{eq:danger_function_threat}
\end{equation}
где:
$\gamma > 0$ — интенсивность роста опасности, связанная с угрозой $\tau$;
$r_\tau$ — радиус влияния угрозы $\tau$.

Таким образом, значение функции опасности увеличивается в моменте, когда угроза находится рядом, и прекращает расти, как только угроза удаляется.

\textbf{8. Задача агентов}.  
Задача агентов заключается в следующем:  
\begin{itemize}
	\item минимизировать функцию общей опасности (\ref{eq:total_threat}):
	\begin{equation}
	U_{\text{total}}(t) = \int_{\mathcal{N}} U(p, t) \, dp;
	\label{eq:total_threat}
	\end{equation}
	\item обнаруживать угрозы $\tau_j \in \mathcal{T}_{\text{threat}}$ и предотвращать их достижение критических точек $\mathcal{C}$.
\end{itemize}

Если угроза $\tau_j$ обнаружена агентом $a_i$, то остальные агенты $a_{k \neq i}$ координируют свои действия для перехвата угрозы, чтобы нейтрализовать ее и предотвратить достижение критической области.

\chapter{Обзор методов координации агентов в многоагентных системах}

\iffalse
В разделе приведена разработанная классификация методов координации агентов, а также описаны известные методы применительно к рассматриваемой задаче~\cite{overview-of-recent}.
\fi

\section{Классификация методов координации агентов в многоагентных системах}

В данной работе классификация методов проводится с учетом особенностей задачи визуального покрытия критических областей, а также требований к моделированию в игровых приложениях~\cite{mas-game-engine}. Для описания методов предлагается следующая классификация:

\subsubsection*{1. Тип взаимодействия между агентами}

Методы координации могут быть разделены на централизованные и децентрализованные~\cite{ieee-mas}:
\begin{itemize}
	\item \textbf{Централизованные методы}: предполагают наличие центрального узла, который координирует действия всех агентов.
	Такие методы обеспечивают глобальную оптимальность решений, но требуют глобальной видимости среды и ее полный анализ;
	\item \textbf{Децентрализованные методы}: агенты принимают решения только на основе полученной ими информации и действуют независимо от других.
	Эти методы более устойчивы к сбоям и масштабируемы, но могут приводить к субоптимальным решениям.
\end{itemize}

\subsubsection*{2. Область восприятия агента}

Методы различаются по тому, какую часть среды может учитывать агент при принятии решений~\cite{ieee-mas}:
\begin{itemize}
	\item \textbf{Методы с глобальным восприятием}: агенты имеют доступ к информации обо всей среде, включая местоположение всех критических точек, других агентов и угроз.
	Это требует высокой вычислительной мощности и связности сети.  
	\item \textbf{Методы с локальным восприятием}: решения принимаются на основе информации из ограниченной области вокруг агента.
	Это снижает требования к вычислительным ресурсам, но может увеличить риск неполного охвата критических точек.
\end{itemize}

\subsubsection*{3. Способ распределения задач между агентами}

Эффективность координации зависит от способа распределения задач~\cite{role-based}:
\begin{itemize}
	\item \textbf{Жестко распределенные задачи}: каждому агенту заранее назначается определенная область или роль, что упрощает координацию, но снижает адаптивность к изменяющимся условиям.
	\item \textbf{Динамическое распределение задач}: задачи перераспределяются в процессе выполнения на основе текущей информации о среде.
	Этот подход обеспечивает большую гибкость, но требует дополнительных вычислений.
\end{itemize}

\subsubsection*{4. Сложность алгоритмов реализации}

Хотя сложность конкретных алгоритмов реализации методов координации агентов может различаться, для сравнения методов полезной является приблизительная оценка трудоёмкости вычисления методов в зависимости от основных параметров модели:
\begin{itemize}
	\item $a$ — количество агентов в системе. Это определяет степень взаимодействия между агентами и влияет на необходимость синхронизации и перерасчёта их состояний.
	\item $v$ — количество вершин графа среды. Вершины представляют возможные положения агентов и критических точек.
	\item $e$ — количество рёбер графа среды, связанное с числом вершин соотношением $e \sim c \cdot v$, где $c$ — среднее число связей для каждой вершины.
	\item $b$ — количество препятствий, влияющих на построение областей видимости и маршрутов.
	\item $t$ — количество угроз -- вражеских объектов, которые нужно обнаружить и нейтрализовать.
	\item $c$ — количество критических областей, требующих визуального контроля.
\end{itemize}

Сводя эти параметры, можно выделить минимально необходимые переменные для оценки сложности: 
\begin{itemize}
	\item $a$ — количество агентов;
	\item $v$ — количество вершин графа среды;
	\item $c$ - количество критических областей;
	\item $b$ — количество препятствий;
	\item $t$ — количество угроз.
\end{itemize}

Оценка трудоемкости производится в нотации <<О>> большое для сравнения асимптотического поведения функций.

\subsubsection*{5. Гибкость метода к изменяющимся условиям}

Из-за того, что среда является высоко динамичной за счет наличия движущихся агентов и изменяющегося уровня опасности, методы могут быть классифицированы по гибкости к изменяющимся условиям следующим образом:
\begin{itemize}
	\item \textbf{Методы с низкой гибкостью}: эффективно работают в статических средах, но требуют значительного времени для перерасчета при изменении условий.  
	\item \textbf{Методы с высокой гибкостью}: автоматически корректируют действия агентов в ответ на изменения в среде, что делает их подходящими для задач в динамических игровых приложениях.
\end{itemize}

\subsubsection*{6. Визуальная правдоподобность движений агентов}

Для игровых приложений важно, чтобы движения агентов выглядели естественно с точки зрения игрока.
Методы могут быть классифицированы по уровню визуальной правдоподобности:
\begin{itemize}
	\item \textbf{Прямолинейные методы}: движения агентов строго следуют оптимальной траектории.
	Это может быть эффективно с точки зрения минимизации затрат, но выглядит механистично и неестественно.  
	\item \textbf{Методы с естественным поведением}: включают элементы непредсказуемости или плавности в траекториях агентов, что улучшает восприятие их действий игроком.
\end{itemize}

\section{Методы координации агентов в многоагентных системах}

\subsection{Метод потенциальных полей}
Метод потенциальных полей основывается на вычислении градиента искусственного потенциала, который направляет движение агентов~\cite{rts-potential-fields}.
Для применения к нашей задаче метод должен учитывать следующие элементы: препятствия, критические области, угрозы, а также общую функцию опасности.

\subsubsection*{Общее описание метода потенциалов}
Потенциал для агента $a_i$ задается как функция (\ref{eq:potential}):
\begin{equation}
	\label{eq:potential}
	V(p_i) = V_{\text{аттрактор}}(p_i) + V_{\text{репеллент}}(p_i),
\end{equation}
где $p_i$ — позиция агента.  
Компоненты потенциала определяются следующим образом:  
\begin{itemize}
	\item $V_{\text{акттрактор}}(p_i)$ — компонент, притягивающий агента к целевым областям (например, критическим точкам).
	\item $V_{\text{репеллент}}(p_i)$ — компонент, отталкивающий агента от препятствий, других агентов и областей с высоким уровнем опасности.
\end{itemize}

Градиент потенциала (\ref{eq:potential_gradient}) определяет направление движения агента:
\begin{equation}
	\label{eq:potential_gradient}
	\dot{p}_i = -\nabla V(p_i),
\end{equation}
где $\dot{p}_i$ — скорость агента.

\subsubsection*{Адаптация метода к задаче}
Для нашей задачи потенциал должен учитывать:
\begin{itemize}[leftmargin=1.6\parindent]
	\item Привлечение агентов к критическим точкам и областям с высокой функцией опасности.
	\item Отталкивание агентов от препятствий и других агентов.
	\item Отталкивание агентов от областей с высокой плотностью угроз.
\end{itemize}

Потенциал определяется по формуле (\ref{eq:potential_final}):
\begin{equation}
	\label{eq:potential_final}
	V(p_i) = \sum_{k=1}^{c} w_k \cdot V_{\text{кр}}(p_i, q_k) + \sum_{j=1}^{b} w_j \cdot V_{\text{преп}}(p_i, o_j) + \sum_{\tau=1}^{t} w_\tau \cdot V_{\text{угр}}(p_i, \tau),
\end{equation}
где:  
\begin{itemize}[leftmargin=1.6\parindent]
	\item $V_{\text{кр}}(p_i, q_k)$ — аттрактор (\ref{eq:potential_attr}), притягивающий $a_i$ к критической точке $q_k$.
	\item $V_{\text{преп}}(p_i, o_j)$ — репеллент (\ref{eq:potential_blocks}), отталкивающий $a_i$ от препятствия $o_j$.
	\item $V_{\text{угр}}(p_i, \tau)$ — репеллент (\ref{eq:potential_threats}), отталкивающий $a_i$ от угрозы $\tau$.
	\item $w_k, w_j, w_\tau$ — весовые коэффициенты.
\end{itemize}

Каждая компонента определяется как:
\begin{align}
	V_{\text{кр}}(p_i, q_k) &= -\frac{1}{\|p_i - q_k\| + \epsilon}, \label{eq:potential_attr} \\
	V_{\text{преп}}(p_i, o_j) &= \frac{1}{\|p_i - o_j\|^2 + \epsilon}, \label{eq:potential_blocks} \\
	V_{\text{угр}}(p_i, \tau) &= \frac{1}{\|p_i - \tau\|^2 + \epsilon}, \label{eq:potential_threat}
\end{align}
где $\epsilon > 0$ предотвращает деление на ноль.

\subsubsection*{Алгоритмическая сложность метода}
Сложность метода определяется трудоемкостью вычислений потенциала для каждого агента. 

Требуется учесть каждую критическую точку $q_k$, препятствие $o_j$ и угрозу $\tau$, а также влияние функции опасности.

Итоговая сложность для одного агента характеризуется формулой~(\ref{eq:potential_complexity_one}):
\begin{equation}
	\label{eq:potential_complexity_one}
	O_{\text{агент}} = O(c + b + t).
\end{equation}

Суммарная сложность для всех агентов характеризуется формулой~(\ref{eq:potential_complexity}):
\begin{equation}
	\label{eq:potential_complexity}
	O_{\text{общая}} = O(a \cdot (c + b + t)).
\end{equation}

\subsubsection*{Классификация метода}
\textbf{Тип взаимодействия:}  
Метод является \textbf{децентрализованным}, так как каждый агент принимает решения на основе локальных вычислений потенциала.  

\textbf{Область восприятия:}  
Метод использует \textbf{локальное восприятие}, ограниченное областью действия потенциала~\cite{bvp-planning}.

\textbf{Распределение задач:}  
Задачи распределяются \textbf{динамически} в процессе вычисления градиента.  

\textbf{Сложность:}  
Итоговая сложность $O(a \cdot (c + b + t))$ является линейной относительно числа агентов $a$ и элементов среды.
Это позволяет применять метод в режиме реального времени.  

\textbf{Гибкость:}  
Метод обладает \textbf{высокой адаптивностью}, так как параметры потенциалов можно изменять в зависимости от текущих условий.

\textbf{Правдоподобность:}  
Метод обеспечивает \textbf{высокий уровень правдоподобности} при тщательной настройке параметров, что показано в \cite{rts-potential-fields}.

\subsection{Метод ролей}

Метод ролей основывается на назначении фиксированных функций или ролей агентам, которые определяют их поведение и задачи в системе~\cite{role-based}.

Для применения к нашей задаче данный метод должен учитывать распределение агентов по функциям патрулирования, защиты критических областей и нейтрализации угроз.

\subsubsection*{Общее описание метода ролей}
В методе ролей каждому агенту $a_i$ назначается роль $r_i$ из множества допустимых ролей $R$, определяемого формулой (\ref{eq:roles}):
\begin{equation}
	\label{eq:roles}
	r_i \in R = \{\text{патрулирование}, \text{защита}, \text{перехват}\}.
\end{equation}

Каждая роль имеет свои задачи:
\begin{itemize}
	\item \textbf{Патрулирование:} агент перемещается по маршруту, покрывающему определенную область.
	\item \textbf{Защита:} агент остается вблизи критической точки и контролирует угрозы в ее окружении.
	\item \textbf{Перехват:} агент направляется к обнаруженной угрозе для ее нейтрализации.
\end{itemize}

Назначение ролей может быть статическим (фиксированное распределение) или динамическим (меняется в зависимости от ситуации).
В рассматриваемой задаче применимо \textbf{динамическое распределение}, где роли пересматриваются в реальном времени на основе состояния среды.

\subsubsection*{Адаптация метода к задаче}
Адаптация метода ролей требует учета следующих факторов:
\begin{itemize}
	\item Выбор ролей агентов в зависимости от текущего уровня опасности $\mathcal{U}(x, y, t)$, положения критических точек и угроз.
	\item Оптимизация распределения ролей для минимизации времени реакции на угрозы и покрытия областей.
\end{itemize}

Процесс распределения ролей описывается формулой (\ref{eq:role_assignment}):
\begin{equation}
	r_i = \arg \min_{r \in R} C(r, p_i, \mathcal{U}, T),
	\label{eq:role_assignment}
\end{equation}
где $C(r, p_i, \mathcal{U}, T)$ — функция стоимости назначения роли $r$ агенту $a_i$, зависящая от его положения $p_i$, функции опасности $\mathcal{U}$ и текущего набора угроз $T$.

Функция стоимости для ролей определяется формулами~(\ref{eq:role_patrol})-(\ref{eq:role_catch}):
\begin{align}
	C_{\text{патрулирование}} &= \sum_{(x, y) \in A_i} \mathcal{U}(x, y, t), \label{eq:role_patrol} \\
	C_{\text{защита}} &= \sum_{q_k \in Q} \frac{1}{\|p_i - q_k\| + \epsilon}, \label{eq:role_defend} \\
	C_{\text{перехват}} &= \min_{\tau \in T} \frac{\|p_i - \tau\|}{v_i}, \label{eq:role_catch}
\end{align}
где:
\begin{itemize}[leftmargin=1.6\parindent]
	\item $A_i$ — область, закрепленная за агентом $a_i$ для патрулирования.
	\item $Q$ — множество критических точек.
	\item $v_i$ — скорость агента $a_i$.
\end{itemize}

\subsubsection*{Алгоритмическая сложность метода}
Рассмотрим сложность распределения ролей.  
\begin{itemize}
	\item Для патрулирования необходимо вычислить сумму значений функции опасности по закрепленной области $A_i$, что требует $O(v)$ операций (по числу вершин графа).
	\item Для защиты требуется рассчитать расстояние до всех критических точек, что требует $O(c)$ операций.
	\item Для перехвата необходимо вычислить расстояние до всех угроз, что требует $O(t)$ операций.
\end{itemize}

Итоговая сложность распределения ролей для одного агента характеризуется формулой (\ref{eq:role_complexity_one}):
\begin{equation}
	\label{eq:role_complexity_one}
	O_{\text{агент}} = O(v + c + t).
\end{equation}

Суммарная сложность для всех агентов оцененивается формулой~(\ref{eq:role_complexity}):
\begin{equation}
	\label{eq:role_complexity}
	O_{\text{общая}} = O(a \cdot (v + c + t)).
\end{equation}

\subsubsection*{Классификация метода}
\textbf{Тип взаимодействия:}  
Метод является \textbf{централизованным}, так как роли назначаются всем агентам на основе глобальных данных о среде.

\textbf{Область восприятия:}  
Метод использует \textbf{глобальное восприятие}, так как распределение ролей требует информации о всей системе.  

\textbf{Распределение задач:}  
Распределение задач является \textbf{динамическим}, так как роли пересматриваются на основе текущего состояния среды.  

\textbf{Сложность:}  
Сложность $O(a \cdot (v + c + t))$ является линейной относительно числа агентов $a$, что позволяет использовать метод в режиме реального времени, если количество вершин $v$ остается не слишком большим.  

\textbf{Гибкость:}  
Метод обладает \textbf{высокой гибкостью}, так как роли могут быть адаптированы к изменениям в среде.  

\textbf{Правдоподобность:}  
Метод обеспечивает \textbf{низкий уровень правдоподобности}, так как агенты обладают ограниченным количеством паттернов поведения, которое задается числом ролей.

\subsection{Метод роя частиц}

Метод роя -- это метод роевого интеллекта, используемый для оптимизации сложных нелинейных целевых функций, основанный на итеративном улучшении решения-кандидата с учетом заданного показателя качества~\cite{particle-swarm}.

\subsubsection*{Общее описание метода роя}
Поведение каждого агента $a_i$ определяется рядом локальных правил:  

1. \textbf{Притяжение:} движение к центру масс соседних агентов описывается формулой (\ref{eq:swarm_attract}):  
\begin{equation}
	\label{eq:swarm_attract}
	f_{\text{притяжение}}(p_i) = k_{\text{пр}} \cdot \left( \frac{1}{|N_i|} \sum_{a_j \in N_i} p_j - p_i \right),
\end{equation}
где $N_i$ — множество агентов в радиусе восприятия $r_{\text{воспр}}$, $k_{\text{пр}}$ — коэффициент притяжения.

2. \textbf{Избегание:} отталкивание от слишком близко расположенных агентов описывается формулой (\ref{eq:swarm_discard}):  
\begin{equation}
	\label{eq:swarm_discard}
	f_{\text{избегание}}(p_i) = \sum_{a_j \in N_i^{\text{близ}}} k_{\text{изб}} \cdot \frac{p_i - p_j}{\|p_i - p_j\|^3},
\end{equation}
где $N_i^{\text{близ}}$ — множество агентов в пределах малого радиуса $r_{\text{мин}}$, $k_{\text{изб}}$ — коэффициент избегания.

3. \textbf{Выравнивание:} согласование направления движения описывается формулой (\ref{eq:swarm_smooth}):  
\begin{equation}
	\label{eq:swarm_smooth}
	f_{\text{выравнивание}}(p_i) = k_{\text{выр}} \cdot \left( \frac{1}{|N_i|} \sum_{a_j \in N_i} \dot{p}_j - \dot{p}_i \right),
\end{equation}
где $k_{\text{выр}}$ — коэффициент выравнивания.

Суммарное движение агента может быть описано формулой (\ref{eq:swarm_total}):
\begin{equation}
	\label{eq:swarm_total}
	\dot{p}_i = f_{\text{притяжение}}(p_i) + f_{\text{избегание}}(p_i) + f_{\text{выравнивание}}(p_i).
\end{equation}

\subsubsection*{Адаптация метода к задаче}
Для нашей задачи метод роя модифицируется следующим образом:  
1. \textbf{Учет функции опасности $\mathcal{U}(x, y, t)$}: агенты перемещаются в области, где $\mathcal{U}$ имеет высокие значения.
Это обеспечивается добавлением аттрактора (\ref{eq:swarm_attractor}):
\begin{equation}
	\label{eq:swarm_attractor}
	f_{\text{опасность}}(p_i) = -k_{\text{оп}} \cdot \nabla \mathcal{U}(p_i, t),
\end{equation}
где $k_{\text{оп}}$ — коэффициент чувствительности к опасности.

2. \textbf{Нейтрализация угроз}: агенты в области видимости угроз $\tau$ перенаправляются к ним.
Дополнительное движение определяется как (\ref{eq:swarm_threat}):
\begin{equation}
	\label{eq:swarm_threat}
	f_{\text{угроза}}(p_i) = -k_{\text{угр}} \cdot \sum_{\tau \in T_{\text{вид}}} \frac{p_i - \tau}{\|p_i - \tau\|^3},
\end{equation}
где $T_{\text{вид}}$ — множество угроз в радиусе видимости агента.

3. \textbf{Балансирование покрытия и плотности}: для предотвращения скопления агентов используется штраф за высокую плотность (\ref{eq:swarm_balance}):
\begin{equation}
	\label{eq:swarm_balance}
	f_{\text{разрежение}}(p_i) = k_{\text{разр}} \cdot \left( \frac{1}{|N_i|} - \rho_{\text{целевая}} \right),
\end{equation}
где $\rho_{\text{целевая}}$ — целевая плотность агентов.

Итоговое движение агента описывается формулой (\ref{eq:swarm_final}):
\begin{equation}
	\label{eq:swarm_final}
	\dot{p}_i^{total} =\dot{p}_i + f_{\text{опасность}}(p_i) + f_{\text{угроза}}(p_i) + f_{\text{разрежение}}(p_i).
\end{equation}

\subsubsection*{Алгоритмическая сложность метода}
Сложность метода роя определяется числом соседей каждого агента.
Обозначим среднее число соседей как $|N_i|$.

\begin{itemize}
	\item На вычисление взаимодействий для одного агента требуется $O(|N_i|)$.
	\item Учет функции опасности требует $O(v)$ операций для каждого агента, так как $\mathcal{U}$ задается на графе.
	\item Для обнаружения угроз в радиусе видимости необходимо $O(t)$.
\end{itemize}

Итоговая сложность для одного агента характеризуется формулой~(\ref{eq:swarm_complexity_one}):
\begin{equation}
	\label{eq:swarm_complexity_one}
	O_{\text{агент}} = O(|N_i| + v + t).
\end{equation}

Общая сложность характеризуется формулой~(\ref{eq:swarm_complexity}):
\begin{equation}
	\label{eq:swarm_complexity}
	O_{\text{общая}} = O(a \cdot (|N_i| + v + t)).
\end{equation}

При фиксированном радиусе восприятия $r_{\text{воспр}}$, трудоемкость остается практически постоянной, что делает метод подходящим для реального времени.

\subsubsection*{Классификация метода}
\textbf{Тип взаимодействия:}  
Метод является \textbf{децентрализованным}, так как агенты принимают решения на основе локальной информации~\cite{swarm-based-coord}.  

\textbf{Область восприятия:}  
Метод использует \textbf{локальное восприятие}, ограниченное радиусом $r_{\text{воспр}}$.  

\textbf{Распределение задач:}  
Задачи распределяются \textbf{динамически} на основе взаимодействия с функцией опасности и угрозами.  

\textbf{Сложность:}  
Сложность $O(a \cdot (|N_i| + v + t))$ линейна относительно числа агентов $a$ и остается практически константной относительно $|N_i|$, что подходит для игр в реальном времени.  

\textbf{Гибкость:}  
Метод обладает \textbf{низкой гибкостью}, так как поведения агентов в основном продиктовано необходимостью группироваться и повторять действия соседних агентов, реагирующих на локальные изменения среды. 

\textbf{Правдоподобность:}
Метод обеспечивает \textbf{высокий уровень правдоподобности}, так как движение агентов обладает свойствами естественного поведения роя~\cite{swarm-based-coord}.

\subsection{Метод планирования на основе теории игр}

Метод планирования на основе теории игр предполагает, что агенты взаимодействуют, решая задачи оптимального поведения в многоагентной среде через формирование и решение математической модели игры~\cite{gurevich2005multiagent}.

\subsubsection*{Общее описание метода}
Модель задачи представляется как стратегическая игра $(A, U)$, где:  
\begin{itemize}
	\item $A = \{A_1, A_2, \dots, A_a\}$ — множество агентов;
	\item $S_i$ — множество стратегий $i$-го агента;
	\item $U_i: S_1 \times S_2 \times \dots \times S_a \to \mathbb{R}$ — функция выигрыша $i$-го агента, зависящая от стратегий всех агентов.
\end{itemize}

В ходе игры каждый агент выбирает стратегию $s_i \in S_i$, стремясь максимизировать свою функцию выигрыша $U_i$~\cite{parsons2002gametheory}.
Решение задачи игры определяется через нахождение равновесий, например, равновесия Нэша, которые удовлетворяют условию (\ref{eq:nash_condition}):  
\begin{equation}
	\label{eq:nash_condition}
	U_i(s_{-i}^*, s_i^*) \geq U_i(s_{-i}^*, s_i) \quad \forall s_i \in S_i,
\end{equation}
где $s_{-i}^*$ — стратегии всех агентов, кроме $i$, в равновесии.

\subsubsection*{Адаптация метода к задаче}
Для нашей задачи метод планирования на основе теории игр модифицируется следующим образом:  

1. \textbf{Множество стратегий}:
Каждый агент выбирает маршрут и целевую область покрытия.
Множество стратегий $S_i$ для агента $i$ включает все возможные пути вдоль графа среды, ведущие к областям покрытия.  

2. \textbf{Функция выигрыша}:  
Функция выигрыша $U_i$ определена как~(\ref{eq:win_function}):
\begin{equation}
	\label{eq:win_function}
	U_i(s_i, s_{-i}) = - \alpha \mathcal{U}(p_i) - \beta \sum_{c_k \in C} \mathcal{U}(c_k) + \gamma \sum_{\tau \in T} d(p_i, \tau),
\end{equation}
где:
\begin{itemize}
	\item $\mathcal{U}(p_i)$ — значение функции опасности в целевом положении $p_i$ агента;
	\item $\mathcal{U}(c_k)$ — значение функции опасности в критической области $c_k$;
	\item $d(p_i, \tau)$ — расстояние до угрозы $\tau$;
	\item $\alpha, \beta, \gamma$ — веса, задающие приоритеты поведения.
\end{itemize}

3. \textbf{Решение игры}:  
Игра решается в реальном времени через итеративное приближение равновесия Нэша~\cite{parsons2002gametheory}.
Для этого каждый агент оптимизирует свою стратегию $s_i$, исходя из стратегий остальных агентов $s_{-i}$ согласно (\ref{eq:strategy_optimization}).
\begin{equation}
	\label{eq:strategy_optimization}
	s_i^* = \arg\max_{s_i \in S_i} U_i(s_i, s_{-i}^*).
\end{equation}

4. \textbf{Нейтрализация угроз}:  
Если угроза $\tau$ находится в области видимости агента $i$, стратегия агента автоматически переходит к ее преследованию и нейтрализации выбирая стратегию по формуле (\ref{eq:strategy_threat}):
\begin{equation}
	\label{eq:strategy_threat}
	s_i = \arg\min_{s_i \in S_i} d(p_i, \tau).
\end{equation}

\subsubsection*{Алгоритмическая сложность метода}
Сложность метода определяется числом агентов, стратегий и итераций поиска равновесия:
\begin{itemize}
	\item Для каждого агента построение множества стратегий $S_i$ требует $O(v^2)$, при применении алгоритма Дейкстры для построения кратчайших путей.
	\item Оценка функции выигрыша $U_i$ для всех стратегий $S_i$ требует $O(|S_i|)$.
	\item Поиск равновесия итеративным методом (например, методом наивной оптимизации) требует $O(k \cdot a \cdot |S_i|)$, где $k$ — число итераций до сходимости.
\end{itemize}

При этом можно принять, что $|S_i| \leq v^2$ и $k=const$, тогда итоговая сложность характеризуется формулой (\ref{eq:game_complexity}):
\begin{equation}
	\label{eq:game_complexity}
	O_{\text{общая}} = O(a \cdot v^2).
\end{equation}

\subsubsection*{Классификация метода}
\textbf{Тип взаимодействия:}  
Метод относится к \textbf{централизованному типу}, так как агенты используют информацию о стратегиях, выбранных другими агентами~\cite{gurevich2005multiagent}.

\textbf{Область восприятия:}  
Метод использует \textbf{локальное восприятие}, так как стратегии формируются на основе окружения агента.

\textbf{Распределение задач:}  
Задачи распределяются \textbf{статически}, поскольку стратегия выбирается на основе рассчитаного оптимума~\cite{gametheory-applcation}.

\textbf{Сложность:}  
Метод имеет сложность $O(a \cdot v^2)$, что делает его трудоемким для сред с большим размером локации.

\textbf{Гибкость:}  
Метод обладает \textbf{высокой} гибкостью, поскольку стратегия агента адаптируется на основе его окружения.

\textbf{Правдоподобность:}  
Правдоподобность метода \textbf{высокая}, так как поведение агентов, основанное на игровых стратегиях, соответствует ожиданиям от разумной координации~\cite{gametheory-applcation}.

\subsection{Метод на основе обучения с подкреплением}

\subsubsection*{Общее описание метода}
Метод Q-обучения относится к виду методов обучения с подкреплением.
Данный метод позволяет моделировать поведение агентов как процесс последовательного принятия решений в среде.
Среда представляется в виде марковского процесса принятия решений~\cite{markov-decision-process}, который задается пятеркой $(S, A, P, R, \gamma)$:
\begin{itemize}[leftmargin=1.6\parindent]
	\item $S$ — множество состояний среды,
	\item $A$ — множество возможных действий агента,
	\item $P(s' | s, a)$ — функция переходов между состояниями при выполнении действия $a$,
	\item $R(s, a)$ — функция вознаграждения за выполнение действия $a$ в состоянии $s$,
	\item $\gamma \in [0, 1]$ — коэффициент дисконтирования будущих вознаграждений.
\end{itemize}

Цель агента заключается в максимизации ожидаемой суммарной дисконтированной награды (\ref{eq:qlearn_reward}):
\begin{equation}
	\label{eq:qlearn_reward}
	G_t = \mathbb{E} \left[ \sum_{k=0}^{\infty} \gamma^k R(s_{t+k}, a_{t+k}) \right].
\end{equation}

Агент учится стратегии $\pi(a|s)$, которая задает вероятность выбора действия $a$ в состоянии $s$.
Обучение стратегии происходит на основе значений функции полезности (\ref{eq:qlearn_qfunc}):
\begin{equation}
	\label{eq:qlearn_qfunc}
	Q^\pi(s, a) = \mathbb{E} \left[ \sum_{k=0}^{\infty} \gamma^k R(s_{t+k}, a_{t+k}) \bigg| s_t = s, a_t = a \right].
\end{equation}

\subsubsection*{Адаптация метода к задаче}
Для задачи визуального покрытия критических областей вводится специфическая структура состояния, действий, и функции награды:
\begin{itemize}
	\item \textbf{Состояния ($S$):} Состояние агента включает:
	\begin{itemize}
		\item Текущую позицию $p_i$ и скорость $\dot{p}_i$ агента,
		\item Значения функции опасности $\mathcal{U}(x, y, t)$ в окрестности агента,
		\item Расположение ближайших критических точек и угроз.
	\end{itemize}
	\item \textbf{Действия ($A$):} Агенты могут выбирать движение в одном из $k$ направлений, задаваемых дискретизацией пространства, или оставаться на месте.
	\item \textbf{Функция вознаграждения ($R$):} Вознаграждение определяется функцией (\ref{eq:qlearn_rewardfunc}):
	\begin{equation}
		\label{eq:qlearn_rewardfunc}
		R(s, a) = -\alpha \cdot \mathcal{U}(p_i, t) + \beta \cdot I_{\text{угроза\_нейтрализована}} - \gamma \cdot \mathcal{L}(p_i),
	\end{equation}
	где $\mathcal{L}(p_i)$ — штраф за выход за границы области, $\alpha, \beta, \gamma$ — коэффициенты весов, $I_{\text{угроза\_нейтрализована}}$ — индикатор нейтрализации угрозы.
\end{itemize}

Для обучения стратегии используется симуляция среды: агенты взаимодействуют с функцией опасности $\mathcal{U}(x, y, t)$, перемещаются между вершинами графа и реагируют на появление угроз $\tau$.
Модель среды обновляется согласно описанной динамике.

\subsubsection*{Алгоритмическая сложность метода}
Обучение с подкреплением включает два основных этапа:

1. \textbf{Симуляция среды:} При фиксированном числе агентов $a$ симуляция одного шага занимает $O(a \cdot v)$ операций, так как необходимо обновить функцию $\mathcal{U}$ и обработать поведение каждого агента.

2. \textbf{Обновление стратегии:} Для алгоритма Q-обучения требуется обновление таблицы $Q(s, a)$, что занимает $O(|S| \cdot |A|)$ операций.

При использовании Deep Q-Learning сложность определяется числом параметров нейронной сети $n_{\text{параметры}}$.

Значения $|S|$, $|A|$, $n_{\text{параметры}}$ фиксированы, так что итоговая сложность обучения характеризуется формулой (\ref{eq:qlearn_complexity}):
\begin{equation}
	\label{eq:qlearn_complexity}
	O_{\text{обучение}} = O(N \cdot (a \cdot v + |S| \cdot |A|)),
\end{equation}
где $N$ — число шагов симуляции.

Применение обученной стратегии в реальном времени требует $O(a)$ операций на каждом шаге.

\subsubsection*{Классификация метода}
\textbf{Тип взаимодействия:}  
Метод является \textbf{децентрализованным}, агенты обучаются индивидуально и учитывают окружение через локальное состояние. 

\textbf{Область восприятия:}  
Метод использует \textbf{локальное восприятие}, так как состояние включает информацию об окружении агента и его параметрах.

\textbf{Распределение задач:}  
Распределение задач \textbf{статическое} во время выполнения, так как стратегия фиксируется после этапа обучения~\cite{markov-decision-process}.  

\textbf{Сложность:}  
Обучение требует больших вычислительных ресурсов ($O_{\text{обучение}}$), однако применение стратегии после обучения не требует трудоемких вычислений.

\textbf{Гибкость:}  
Метод обладает \textbf{высокой гибкостью}, так как стратегия может адаптироваться к сложным динамическим сценариям.  

\textbf{Правдоподобность:}  
Метод обеспечивает \textbf{высокую правдоподобность}, так как обученные стратегии могут воспроизводить реалистичное поведение при корректной настройке параметров~\cite{hysteretic-q-learning}.

\chapter{Сравнение методов координации агентов в многоагентных системах}

На основе классификации методов, описанных выше, проведем сравнительный анализ по критериям, сформулированным ранее. В таблицах \ref{tab:comparison1}-\ref{tab:comparison2} приведены ключевые характеристики методов координации.

\begin{table}[H]
	\centering
	\caption{Сравнение методов координации агентов в МАС (часть 1/2)}
	\label{tab:comparison1}
	\begin{adjustbox}{width=\textwidth}
		\begin{tabular}{|c|c|c|c|}
		\hline
		\textbf{Метод} & \makecell{\textbf{Тип}\\\textbf{взаимодействия}} & \makecell{\textbf{Область}\\\textbf{восприятия}} & \makecell{\textbf{Распределение}\\\textbf{задач}} \\ \hline
		Потенциальных полей    & Децентрализованный         & Локальная                  & Динамическое                 \\ \hline
		Ролей          & Централизованный            & Глобальная                 & Динамическое                 \\ \hline
		Роя частиц     & Децентрализованный         & Локальная                  & Динамическое                 \\ \hline
		Теоретико-игровой     & Централизованный           & Локальная                 & Статическое                  \\ \hline
		Обучения с подкреплением & Децентрализованный         & Локальная                 & Статическое                  \\ \hline
		\end{tabular}
	\end{adjustbox}
\end{table}

\begin{table}[H]
	\centering
	\caption{Сравнение методов координации агентов в МАС (часть 2/2)}
	\label{tab:comparison2}
	\begin{adjustbox}{width=\textwidth}
		\begin{tabular}{|c|c|c|c|}
			\hline
			\textbf{Метод} & \makecell{\textbf{Сложность}} & \makecell{\textbf{Гибкость}} & \makecell{\textbf{Правдоподобность}} \\ \hline
			Потенциальных полей    & $O(a \cdot (c + b + t))$   & Высокая                    & Высокая                      \\ \hline
			Ролей          & $O(a \cdot (v + c + t))$   & Высокая                    & Низкая                      \\ \hline
			Роя частиц     & $O(a \cdot (|N_i| + v + t))$ & Низкая                     & Высокая                      \\ \hline
			Теоретико-игровой & $O(a \cdot v^2)$         & Высокая                    & Высокая                      \\ \hline
			Обучения с подкреплением & $O(a)$                  & Высокая                    & Высокая                      \\ \hline
		\end{tabular}
	\end{adjustbox}
\end{table}

\subsubsection*{Вывод}

Исходя из сравнительной таблицы можно сделать следующие выводы:
\begin{enumerate}
	\item Наименее трудоемким для применения является метод обучения с подкреплением, однако он плохо интерпретируем и трудоемок при обучении.
	\item Метод потенциальных полей и теоретико-игровой методы обеспечивают высокую степень гибкости и правдоподобности движения агентов, однако трудоемкость работы алгоритмов, реализующих теоретико-игровой метод выше, чем у метода потенциальных полей.
\end{enumerate}

%\chapter{Обзор методов координации агентов в многоагентных системах}

\section{Классификация методов координации агентов в многоагентных системах}

Классификация методов координации агентов в многоагентных системах (МАС) позволяет структурировать подходы к решению задач, связанных с организацией взаимодействия агентов.
В данной работе классификация методов проводится с учётом особенностей задачи визуального покрытия критических областей, а также требований к моделированию в игровых приложениях. Для описания методов предлагается следующая классификация:

\subsubsection*{1. Тип взаимодействия между агентами}
Методы координации могут быть разделены на централизованные и децентрализованные:
\begin{itemize}
	\item \textbf{Централизованные методы}: предполагают наличие центрального узла (контроллера), который координирует действия всех агентов.
	Такие методы обеспечивают глобальную оптимальность решений, но увеличивают нагрузку на центральный процессор и могут быть менее устойчивыми к сбоям связи;
	\item \textbf{Децентрализованные методы}: каждое агентное решение принимается на основе локальной информации и взаимодействия с соседними агентами.
	Эти методы более устойчивы к сбоям и масштабируемы, но могут приводить к субоптимальным решениям в глобальной перспективе.
\end{itemize}

\subsubsection*{2. Область восприятия агента}
Методы различаются по тому, какую часть среды может учитывать агент при принятии решений:
\begin{itemize}
	\item \textbf{Методы с глобальным восприятием}: агенты имеют доступ к информации обо всей среде, включая местоположение всех критических точек, других агентов и угроз.
	Это требует высокой вычислительной мощности и связности сети.  
	\item \textbf{Методы с локальным восприятием}: решения принимаются на основе информации из ограниченной области вокруг агента.
	Это снижает требования к вычислительным ресурсам, но может увеличить риск неполного охвата критических точек.
\end{itemize}

\subsubsection*{3. Способ распределения задач между агентами}
Эффективность координации зависит от способа распределения задач:
\begin{itemize}
	\item \textbf{Жёстко распределённые задачи}: каждому агенту заранее назначается определённая область или роль, что упрощает координацию, но снижает адаптивность к изменяющимся условиям.
	\item \textbf{Динамическое распределение задач}: задачи перераспределяются в процессе выполнения на основе текущей информации о среде.
	Этот подход обеспечивает большую гибкость, но требует дополнительных вычислений.
\end{itemize}

\subsubsection*{4. Реализационная сложность алгоритма}
Классификация методов должна учитывать их вычислительную сложность:
\begin{itemize}
	\item \textbf{Алгоритмы с полиномиальной сложностью}: подходят для применения в игровых приложениях, где требуется работа в режиме мягкого реального времени.  
	\item \textbf{Алгоритмы с экспоненциальной сложностью}: могут использоваться для моделирования небольших систем или предварительных расчётов, но не пригодны для динамического применения в игре.
\end{itemize}
Анализ сложности должен учитывать как время вычислений, так и объём памяти, необходимый для хранения информации о среде и агентных состояниях.

\subsubsection*{5. Гибкость метода к изменяющимся условиям}
Задача визуального контроля критических областей в играх требует учёта изменяющейся среды и появления новых угроз.
В связи с этим методы могут быть классифицированы следующим образом:
\begin{itemize}
	\item \textbf{Методы с низкой адаптивностью}: эффективно работают в статических средах, но требуют значительного времени для перерасчёта при изменении условий.  
	\item \textbf{Методы с высокой адаптивностью}: автоматически корректируют действия агентов в ответ на изменения в среде, что делает их подходящими для задач в динамических игровых приложениях.
\end{itemize}

\subsubsection*{6. Визуальная правдоподобность движений агентов}
Для игровых приложений важно, чтобы движения агентов выглядели естественно с точки зрения игрока.
Методы могут быть классифицированы по уровню визуальной правдоподобности:
\begin{itemize}
	\item \textbf{Прямолинейные методы}: движения агентов строго следуют оптимальной траектории.
	Это может быть эффективно с точки зрения минимизации затрат, но выглядит механистично и неестественно.  
	\item \textbf{Методы с естественным поведением}: включают элементы непредсказуемости или плавности в траекториях агентов, что улучшает восприятие их действий игроком.
\end{itemize}

\section{Методы координации агентов в многоагентных системах}

В данном разделе рассматриваются основные методы координации агентов, которые могут быть использованы для решения задачи визуального контроля критических областей. Каждый метод кратко описан, а затем классифицирован по критериям, представленным в предыдущем разделе.

\subsection{Метод потенциалов}
Метод потенциалов основывается на вычислении градиента искусственного потенциала, который направляет движение агентов.
Для применения к нашей задаче метод должен учитывать следующие элементы: препятствия, критические области, угрозы, а также общую функцию опасности.

\subsubsection*{Общее описание метода потенциалов}
Потенциал для агента $a_i$ задаётся как функция:
\begin{equation}
	V(p_i) = V_{\text{аттракция}}(p_i) + V_{\text{репеллент}}(p_i),
\end{equation}
где $p_i$ — позиция агента.  
Компоненты потенциала определяются следующим образом:  
\begin{itemize}
	\item $V_{\text{аттракция}}(p_i)$ — компонент, притягивающий агента к целевым областям (например, критическим точкам).
	\item $V_{\text{репеллент}}(p_i)$ — компонент, отталкивающий агента от препятствий, других агентов и областей с высоким уровнем опасности.
\end{itemize}

Градиент потенциала $\nabla V(p_i)$ определяет направление движения агента:
\begin{equation}
	\dot{p}_i = -\nabla V(p_i),
\end{equation}
где $\dot{p}_i$ — скорость агента.

\subsubsection*{Адаптация метода к задаче}
Для нашей задачи потенциал должен учитывать:
\begin{itemize}
	\item Привлечение агентов к критическим точкам и областям с высокой функцией опасности.
	\item Отталкивание агентов от препятствий и других агентов.
	\item Отталкивание агентов от областей с высокой плотностью угроз.
\end{itemize}

Потенциал определяется как:
\begin{equation}
	V(p_i) = \sum_{k=1}^{c} w_k \cdot V_{\text{кр}}(p_i, q_k) + \sum_{j=1}^{b} w_j \cdot V_{\text{преп}}(p_i, o_j) + \sum_{\tau=1}^{t} w_\tau \cdot V_{\text{угр}}(p_i, \tau),
\end{equation}
где:  
\begin{itemize}
	\item $V_{\text{кр}}(p_i, q_k)$ — аттрактор, притягивающий агента $a_i$ к критической точке $q_k$.
	\item $V_{\text{преп}}(p_i, o_j)$ — репеллент, отталкивающий агента $a_i$ от препятствия $o_j$.
	\item $V_{\text{угр}}(p_i, \tau)$ — репеллент, отталкивающий агента $a_i$ от угрозы $\tau$.
	\item $w_k, w_j, w_\tau$ — весовые коэффициенты.
\end{itemize}

Каждая компонента определяется как:
\begin{align}
	V_{\text{кр}}(p_i, q_k) &= -\frac{1}{\|p_i - q_k\| + \epsilon}, \\
	V_{\text{преп}}(p_i, o_j) &= \frac{1}{\|p_i - o_j\|^2 + \epsilon}, \\
	V_{\text{угр}}(p_i, \tau) &= \frac{1}{\|p_i - \tau\|^2 + \epsilon},
\end{align}
где $\epsilon > 0$ предотвращает деление на ноль.

\subsubsection*{Алгоритмическая сложность метода}
Сложность метода определяется количеством вычислений потенциала для каждого агента.  
\begin{itemize}
	\item Для каждой критической точки $q_k$ требуется $O(c)$ вычислений потенциала.
	\item Для каждого препятствия $o_j$ требуется $O(b)$ вычислений.
	\item Для каждой угрозы $\tau$ требуется $O(t)$ вычислений.
\end{itemize}
Итоговая сложность для одного агента:
\begin{equation}
	O_{\text{агент}} = O(c + b + t).
\end{equation}
Суммарная сложность для всех агентов:
\begin{equation}
	O_{\text{общая}} = O(a \cdot (c + b + t)).
\end{equation}

\subsubsection*{Классификация метода}
\textbf{Тип взаимодействия:}  
Метод является \textbf{децентрализованным}, так как каждый агент принимает решения на основе локальных вычислений потенциала.  

\textbf{Область восприятия:}  
Метод использует \textbf{локальное восприятие}, ограниченное областью действия потенциала.  

\textbf{Распределение задач:}  
Задачи распределяются \textbf{динамически} в процессе вычисления градиента.  

\textbf{Сложность:}  
Итоговая сложность $O(a \cdot (c + b + t))$ является линейной относительно числа агентов $a$ и элементов среды.
Это позволяет применять метод в режиме реального времени.  

\textbf{Гибкость:}  
Метод обладает \textbf{высокой адаптивностью}, так как параметры потенциалов можно изменять в зависимости от текущих условий.  

\textbf{Правдоподобность:}  
Метод обеспечивает \textbf{средний уровень правдоподобности}, так как движение агентов может выглядеть механистично из-за прямолинейного следования градиенту.

\subsection{Метод ролей}

Метод ролей основывается на назначении фиксированных функций или ролей агентам, которые определяют их поведение и задачи в системе.
Для применения к нашей задаче данный метод должен учитывать распределение агентов по функциям патрулирования, защиты критических областей и нейтрализации угроз.

\subsubsection*{Общее описание метода ролей}
В методе ролей каждому агенту $a_i$ назначается роль $r_i$ из множества допустимых ролей $R$:
\begin{equation}
	r_i \in R = \{\text{патрулирование}, \text{защита}, \text{перехват}\}.
\end{equation}

Каждая роль имеет уникальные задачи:
\begin{itemize}
	\item \textbf{Патрулирование:} агент перемещается по маршруту, покрывающему определённую область.
	\item \textbf{Защита:} агент остаётся вблизи критической точки и контролирует угрозы в её окружении.
	\item \textbf{Перехват:} агент направляется к обнаруженной угрозе для её нейтрализации.
\end{itemize}

Назначение ролей может быть статическим (фиксированное распределение) или динамическим (меняется в зависимости от ситуации).
В нашей задаче используется \textbf{динамическое распределение}, где роли пересматриваются в реальном времени на основе состояния среды.

\subsubsection*{Адаптация метода к задаче}
Адаптация метода ролей требует учёта следующих факторов:
\begin{itemize}
	\item Выбор ролей агентов в зависимости от текущего уровня опасности $\mathcal{U}(x, y, t)$, положения критических точек и угроз.
	\item Оптимизация распределения ролей для минимизации времени реакции на угрозы и покрытия областей.
\end{itemize}

Процесс распределения ролей описывается следующим образом:
\begin{equation}
	r_i = \arg \min_{r \in R} C(r, p_i, \mathcal{U}, T),
	\label{eq:role_assignment}
\end{equation}
где $C(r, p_i, \mathcal{U}, T)$ — функция стоимости назначения роли $r$ агенту $a_i$, зависящая от его положения $p_i$, функции опасности $\mathcal{U}$ и текущего набора угроз $T$.

Функция стоимости для каждой роли определяется как:
\begin{align}
	C_{\text{патрулирование}} &= \sum_{(x, y) \in A_i} \mathcal{U}(x, y, t), \\
	C_{\text{защита}} &= \sum_{q_k \in Q} \frac{1}{\|p_i - q_k\| + \epsilon}, \\
	C_{\text{перехват}} &= \min_{\tau \in T} \frac{\|p_i - \tau\|}{v_i},
\end{align}
где:
\begin{itemize}
	\item $A_i$ — область, закреплённая за агентом $a_i$ для патрулирования.
	\item $Q$ — множество критических точек.
	\item $v_i$ — скорость агента $a_i$.
\end{itemize}

\subsubsection*{Алгоритмическая сложность метода}
Рассмотрим сложность распределения ролей.  
\begin{itemize}
	\item Для патрулирования необходимо вычислить сумму значений функции опасности по закреплённой области $A_i$, что требует $O(v)$ операций (по числу вершин графа).
	\item Для защиты требуется рассчитать расстояние до всех критических точек, что требует $O(c)$ операций.
	\item Для перехвата необходимо вычислить расстояние до всех угроз, что требует $O(t)$ операций.
\end{itemize}

Итоговая сложность распределения ролей для одного агента:
\begin{equation}
	O_{\text{агент}} = O(v + c + t).
\end{equation}

Суммарная сложность для всех агентов:
\begin{equation}
	O_{\text{общая}} = O(a \cdot (v + c + t)).
\end{equation}

\subsubsection*{Классификация метода}
\textbf{Тип взаимодействия:}  
Метод является \textbf{гибридным}, так как роли могут назначаться централизованно (внешним координатором) или децентрализованно (каждым агентом на основе локальных данных).

\textbf{Область восприятия:}  
Метод использует \textbf{глобальное восприятие}, так как распределение ролей требует информации о всей системе.  

\textbf{Распределение задач:}  
Распределение задач является \textbf{динамическим}, так как роли пересматриваются на основе текущего состояния среды.  

\textbf{Сложность:}  
Сложность $O(a \cdot (v + c + t))$ является линейной относительно числа агентов $a$, что позволяет использовать метод в режиме реального времени, если количество вершин $v$, критических точек $c$ и угроз $t$ остаётся умеренным.  

\textbf{Гибкость:}  
Метод обладает \textbf{высокой гибкостью}, так как роли могут адаптироваться к изменениям в среде.  

\textbf{Правдоподобность:}  
Метод обеспечивает \textbf{высокий уровень правдоподобности}, так как распределение ролей позволяет моделировать координированное поведение агентов.

\subsection{Метод роя}

Метод роя основывается на моделировании поведения агентов как коллективного взаимодействия группы, где каждый агент руководствуется правилами взаимодействия с соседями, формируя поведение, схожее с роем насекомых или стаей птиц.
Данный метод обеспечивает децентрализованную координацию агентов за счёт локальных правил.

\subsubsection*{Общее описание метода роя}
Поведение каждого агента $a_i$ определяется рядом локальных правил:  
1. \textbf{Притяжение:} движение к центру масс соседних агентов:  
\begin{equation}
	f_{\text{притяжение}}(p_i) = k_{\text{пр}} \cdot \left( \frac{1}{|N_i|} \sum_{a_j \in N_i} p_j - p_i \right),
\end{equation}
где $N_i$ — множество агентов в радиусе восприятия $r_{\text{воспр}}$, $k_{\text{пр}}$ — коэффициент притяжения.

2. \textbf{Избегание:} отталкивание от слишком близко расположенных агентов:  
\begin{equation}
	f_{\text{избегание}}(p_i) = \sum_{a_j \in N_i^{\text{близ}}} k_{\text{изб}} \cdot \frac{p_i - p_j}{\|p_i - p_j\|^3},
\end{equation}
где $N_i^{\text{близ}}$ — множество агентов в пределах малого радиуса $r_{\text{мин}}$, $k_{\text{изб}}$ — коэффициент избегания.

3. \textbf{Выравнивание:} согласование направления движения:  
\begin{equation}
	f_{\text{выравнивание}}(p_i) = k_{\text{выр}} \cdot \left( \frac{1}{|N_i|} \sum_{a_j \in N_i} \dot{p}_j - \dot{p}_i \right),
\end{equation}
где $k_{\text{выр}}$ — коэффициент выравнивания.

Суммарное движение агента:
\begin{equation}
	\dot{p}_i = f_{\text{притяжение}}(p_i) + f_{\text{избегание}}(p_i) + f_{\text{выравнивание}}(p_i).
\end{equation}

\subsubsection*{Адаптация метода к задаче}
Для нашей задачи метод роя модифицируется следующим образом:  
1. \textbf{Учёт функции опасности $\mathcal{U}(x, y, t)$}: агенты перемещаются в области, где $\mathcal{U}$ имеет высокие значения.
Это обеспечивается добавлением аттрактора:
\begin{equation}
	f_{\text{опасность}}(p_i) = -k_{\text{оп}} \cdot \nabla \mathcal{U}(p_i, t),
\end{equation}
где $k_{\text{оп}}$ — коэффициент чувствительности к опасности.

2. \textbf{Нейтрализация угроз}: агенты в области видимости угроз $\tau$ перенаправляются к ним.
Дополнительное движение определяется как:
\begin{equation}
	f_{\text{угроза}}(p_i) = -k_{\text{угр}} \cdot \sum_{\tau \in T_{\text{вид}}} \frac{p_i - \tau}{\|p_i - \tau\|^3},
\end{equation}
где $T_{\text{вид}}$ — множество угроз в радиусе видимости агента.

3. \textbf{Балансирование покрытия и плотности}: для предотвращения скопления агентов используется штраф за высокую плотность:
\begin{equation}
	f_{\text{разрежение}}(p_i) = k_{\text{разр}} \cdot \left( \frac{1}{|N_i|} - \rho_{\text{целевая}} \right),
\end{equation}
где $\rho_{\text{целевая}}$ — целевая плотность агентов.

Итоговое движение агента:
\begin{equation}
	\dot{p}_i = f_{\text{притяжение}}(p_i) + f_{\text{избегание}}(p_i) + f_{\text{выравнивание}}(p_i) + f_{\text{опасность}}(p_i) + f_{\text{угроза}}(p_i) + f_{\text{разрежение}}(p_i).
\end{equation}

\subsubsection*{Алгоритмическая сложность метода}
Сложность метода роя определяется числом соседей каждого агента
При фиксированном радиусе восприятия $r_{\text{воспр}}$ среднее число соседей $|N_i| \sim O(\rho \cdot \pi r_{\text{воспр}}^2)$, где $\rho$ — плотность агентов.  

\begin{itemize}
	\item На вычисление взаимодействий для одного агента требуется $O(|N_i|)$.
	\item Учёт функции опасности требует $O(v)$ операций для каждого агента, так как $\mathcal{U}$ задаётся на графе.
	\item Для обнаружения угроз в радиусе видимости необходимо $O(t)$.
\end{itemize}

Итоговая сложность для одного агента:
\begin{equation}
	O_{\text{агент}} = O(|N_i| + v + t).
\end{equation}

Суммарная сложность для всех агентов:
\begin{equation}
	O_{\text{общая}} = O(a \cdot (|N_i| + v + t)).
\end{equation}

При фиксированном радиусе восприятия $r_{\text{воспр}}$, сложность $|N_i|$ остаётся практически постоянной, что делает метод подходящим для реального времени.

\subsubsection*{Классификация метода}
\textbf{Тип взаимодействия:}  
Метод является \textbf{децентрализованным}, так как агенты принимают решения на основе локальной информации.  

\textbf{Область восприятия:}  
Метод использует \textbf{локальное восприятие}, ограниченное радиусом $r_{\text{воспр}}$.  

\textbf{Распределение задач:}  
Задачи распределяются \textbf{динамически} через взаимодействия с функцией опасности и угрозами.  

\textbf{Сложность:}  
Сложность $O(a \cdot (|N_i| + v + t))$ линейна относительно числа агентов $a$ и остаётся практически константной относительно $|N_i|$, что подходит для игр в реальном времени.  

\textbf{Гибкость:}  
Метод обладает \textbf{высокой гибкостью}, так как может адаптироваться к изменяющимся условиям среды.  

\textbf{Правдоподобность:}  
Метод обеспечивает \textbf{высокий уровень правдоподобности}, так как движение агентов напоминает естественное поведение роя или стаи.

\subsection{Метод планирования на основе теории игр}

Метод планирования на основе теории игр предполагает, что агенты взаимодействуют, решая задачи оптимального поведения в многоагентной среде через формирование и решение математической модели игры~\cite{gurevich2005multiagent}.

\subsubsection*{Общее описание метода}
Модель задачи представляется как стратегическая игра $(A, U)$, где:  
\begin{itemize}
	\item $A = \{A_1, A_2, \dots, A_a\}$ — множество агентов;
	\item $S_i$ — множество стратегий $i$-го агента;
	\item $U_i: S_1 \times S_2 \times \dots \times S_a \to \mathbb{R}$ — функция выигрыша (utility) $i$-го агента, зависящая от стратегий всех агентов.
\end{itemize}

В ходе игры каждый агент выбирает стратегию $s_i \in S_i$, стремясь максимизировать свою функцию выигрыша $U_i$.
Решение задачи игры определяется через нахождение равновесий, например, равновесия Нэша, которые удовлетворяют условию:  
\begin{equation}
	U_i(s_{-i}^*, s_i^*) \geq U_i(s_{-i}^*, s_i) \quad \forall s_i \in S_i,
\end{equation}
где $s_{-i}^*$ — стратегии всех агентов, кроме $i$, в равновесии.

\subsubsection*{Адаптация метода к задаче}
Для нашей задачи метод планирования на основе теории игр модифицируется следующим образом:  

1. \textbf{Множество стратегий}:  
Каждый агент выбирает маршрут и целевую область покрытия.
Множество стратегий $S_i$ для агента $i$ включает все возможные пути вдоль графа среды, ведущие к областям покрытия.  

2. \textbf{Функция выигрыша}:  
Функция выигрыша $U_i$ определяется следующими факторами:
\begin{equation}
	U_i(s_i, s_{-i}) = - \alpha \mathcal{U}(p_i) - \beta \sum_{c_k \in C} \mathcal{U}(c_k) + \gamma \sum_{\tau \in T} d(p_i, \tau),
\end{equation}
где:
\begin{itemize}
	\item $\mathcal{U}(p_i)$ — значение функции опасности в текущем положении $p_i$ агента;
	\item $\mathcal{U}(c_k)$ — значение функции опасности в критической области $c_k$;
	\item $d(p_i, \tau)$ — расстояние до ближайшей угрозы $\tau$;
	\item $\alpha, \beta, \gamma$ — веса, задающие приоритеты поведения.
\end{itemize}

3. \textbf{Решение игры}:  
Игра решается в реальном времени через итеративное приближение равновесия Нэша.
Для этого каждый агент оптимизирует свою стратегию $s_i$, исходя из стратегий остальных агентов $s_{-i}$:
\begin{equation}
	s_i^* = \arg\max_{s_i \in S_i} U_i(s_i, s_{-i}^*).
\end{equation}

4. \textbf{Нейтрализация угроз}:  
Если угроза $\tau$ находится в области видимости агента $i$, стратегия агента автоматически переходит к её преследованию и нейтрализации:
\begin{equation}
	s_i = \arg\min_{s_i \in S_i} d(p_i, \tau).
\end{equation}

\subsubsection*{Алгоритмическая сложность метода}
Сложность метода определяется числом агентов, стратегий и итераций поиска равновесия:
\begin{itemize}
	\item Для каждого агента построение множества стратегий $S_i$ требует $O(v + e)$, так как пути задаются графом среды.
	\item Оценка функции выигрыша $U_i$ для всех стратегий $S_i$ требует $O(|S_i|)$.
	\item Поиск равновесия итеративным методом (например, методом наивной оптимизации) требует $O(k \cdot a \cdot |S_i|)$, где $k$ — число итераций до сходимости.
\end{itemize}
Итоговая сложность:
\begin{equation}
	O_{\text{общая}} = O(k \cdot a \cdot (v + e + |S_i|)).
\end{equation}

\subsubsection*{Классификация метода}
\textbf{Тип взаимодействия:}  
Метод относится к \textbf{гибридному типу}, так как может быть как кооперативным (оптимизация глобальной цели), так и некооперативным (оптимизация индивидуальных целей).

\textbf{Область восприятия:}  
Метод использует \textbf{глобальное восприятие}, так как решение игры требует информации о стратегиях всех агентов.

\textbf{Распределение задач:}  
Задачи распределяются \textbf{статически}, поскольку стратегия выбирается на основе заранее рассчитанных оптимумов.

\textbf{Сложность:}  
Метод имеет сложность $O(k \cdot a \cdot (v + e + |S_i|))$, что делает его трудоёмким для больших систем с высоким числом агентов.

\textbf{Гибкость:}  
Гибкость метода \textbf{средняя}, так как стратегия агента может адаптироваться только после пересчёта равновесия.

\textbf{Правдоподобность:}  
Правдоподобность \textbf{высокая}, так как поведение агентов, основанное на игровых стратегиях, соответствует ожиданиям от разумной координации.

\subsection{Метод на основе обучения с подкреплением}

\subsubsection*{Общее описание метода}
Метод обучения с подкреплением (RL) моделирует поведение агентов как процесс последовательного принятия решений в среде.
Среда представляется в виде марковского процесса принятия решений (Markov Decision Process, MDP), который задаётся пятёркой $(S, A, P, R, \gamma)$:
\begin{itemize}
	\item $S$ — множество состояний среды,
	\item $A$ — множество возможных действий агента,
	\item $P(s' | s, a)$ — функция переходов между состояниями при выполнении действия $a$,
	\item $R(s, a)$ — функция вознаграждения за выполнение действия $a$ в состоянии $s$,
	\item $\gamma \in [0, 1]$ — коэффициент дисконтирования будущих вознаграждений.
\end{itemize}

Цель агента заключается в максимизации ожидаемой суммарной дисконтированной награды:
\begin{equation}
	G_t = \mathbb{E} \left[ \sum_{k=0}^{\infty} \gamma^k R(s_{t+k}, a_{t+k}) \right].
\end{equation}

Агент учится стратегии $\pi(a|s)$, которая задаёт вероятность выбора действия $a$ в состоянии $s$.
Обучение стратегии происходит на основе значений функции полезности:
\begin{equation}
	Q^\pi(s, a) = \mathbb{E} \left[ \sum_{k=0}^{\infty} \gamma^k R(s_{t+k}, a_{t+k}) \bigg| s_t = s, a_t = a \right].
\end{equation}

Классическими алгоритмами обучения являются Q-Learning и глубокое Q-обучение (Deep Q-Learning).

\subsubsection*{Адаптация метода к задаче}
Для задачи визуального покрытия критических областей вводится специфическая структура состояния, действий, и функции награды:
\begin{itemize}
	\item \textbf{Состояния ($S$):} Состояние агента включает:
	\begin{itemize}
		\item Текущую позицию $p_i$ и скорость $\dot{p}_i$ агента,
		\item Значения функции опасности $\mathcal{U}(x, y, t)$ в окрестности агента,
		\item Расположение ближайших критических точек и угроз.
	\end{itemize}
	\item \textbf{Действия ($A$):} Агенты могут выбирать движение в одном из $k$ направлений, задаваемых дискретизацией пространства, или оставаться на месте.
	\item \textbf{Функция вознаграждения ($R$):} Вознаграждение определяется следующим образом:
	\begin{equation}
		R(s, a) = -\alpha \cdot \mathcal{U}(p_i, t) + \beta \cdot I_{\text{угроза\_нейтрализована}} - \gamma \cdot \mathcal{L}(p_i),
	\end{equation}
	где $\mathcal{L}(p_i)$ — штраф за выход за границы области, $\alpha, \beta, \gamma$ — коэффициенты весов, $I_{\text{угроза\_нейтрализована}}$ — индикатор нейтрализации угрозы.
\end{itemize}

Для обучения стратегии используется симуляция среды: агенты взаимодействуют с функцией опасности $\mathcal{U}(x, y, t)$, перемещаются между вершинами графа и реагируют на появление угроз $\tau$.
Модель среды обновляется согласно описанной динамике.

\subsubsection*{Алгоритмическая сложность метода}
Обучение с подкреплением включает два основных этапа:

1. \textbf{Симуляция среды:} При фиксированном числе агентов $a$ симуляция одного шага занимает $O(a \cdot v)$ операций, так как необходимо обновить функцию $\mathcal{U}$ и обработать поведение каждого агента.
2. \textbf{Обновление стратегии:} Для алгоритма Q-Learning требуется обновление таблицы $Q(s, a)$, что занимает $O(|S| \cdot |A|)$ операций.
При использовании Deep Q-Learning сложность определяется числом параметров нейронной сети $n_{\text{параметры}}$.

Итоговая сложность обучения:
\begin{equation}
	O_{\text{обучение}} = O(N \cdot (a \cdot v + |S| \cdot |A|)),
\end{equation}
где $N$ — число шагов симуляции.
Для Deep Q-Learning $|S| \cdot |A|$ заменяется на $n_{\text{параметры}}$.

Применение обученной стратегии в реальном времени требует $O(a \cdot n_{\text{параметры}})$ операций на каждом шаге.

\subsubsection*{Классификация метода}
\textbf{Тип взаимодействия:}  
Метод является \textbf{гибридным}: агенты обучаются индивидуально, но стратегия может учитывать глобальные факторы через функцию состояния.  

\textbf{Область восприятия:}  
Метод использует \textbf{глобальное восприятие}, так как состояние может включать информацию о всей среде, например, через значения $\mathcal{U}$ в крупных областях.  

\textbf{Распределение задач:}  
Распределение задач \textbf{статическое} во время выполнения, так как стратегия фиксируется после этапа обучения.  

\textbf{Сложность:}  
Обучение требует больших вычислительных ресурсов ($O_{\text{обучение}}$), однако применение стратегии возможно в реальном времени.  

\textbf{Гибкость:}  
Метод обладает \textbf{высокой гибкостью}, так как стратегия может адаптироваться к сложным динамическим сценариям.  

\textbf{Правдоподобность:}  
Метод обеспечивает \textbf{высокую правдоподобность}, так как обученные стратегии могут воспроизводить реалистичное поведение.
%\chapter{Сравнение методов координации агентов в многоагентных системах}

На основе классификации методов, описанных выше, проведём сравнительный анализ по критериям, сформулированным ранее. Таблица \ref{tab:comparison} суммирует ключевые свойства методов координации.

\begin{landscape}
	\begin{table}[H]
		\centering
		\caption{Сравнительный анализ методов координации агентов в многоагентной системе}
		\label{tab:comparison}
		\begin{tabular}{|c|c|c|c|c|c|c|}
			\hline
			\multicolumn{4}{|c|}{\textbf{Часть 1: Тип взаимодействия, Область восприятия, Распределение задач}} \\ \hline
			\textbf{Метод} & \textbf{Тип взаимодействия} & \textbf{Область восприятия} & \textbf{Распределение задач} \\ \hline
			Потенциалов    & Децентрализованный         & Локальная                  & Динамическое                 \\ \hline
			Ролей          & Гибридный                  & Глобальная                 & Динамическое                 \\ \hline
			Роя            & Децентрализованный         & Локальная                  & Динамическое                 \\ \hline
			Теории игр     & Гибридный                  & Глобальная                 & Статическое                  \\ \hline
			Обучения       & Гибридный                  & Глобальная                 & Статическое                  \\ \hline
			\multicolumn{4}{|c|}{} \\ % Пустая строка для визуального разделения
			\hline
			\multicolumn{4}{|c|}{\textbf{Часть 2: Сложность, Гибкость, Правдоподобность}} \\ \hline
			\textbf{Метод} & \textbf{Сложность}         & \textbf{Гибкость}          & \textbf{Правдоподобность}    \\ \hline
			Потенциалов    & $O(a \cdot (c + b + t))$   & Высокая                    & Средняя                      \\ \hline
			Ролей          & $O(a \cdot (v + c + t))$   & Высокая                    & Высокая                      \\ \hline
			Роя            & $O(a \cdot (|N_i| + v + t))$ & Высокая                    & Высокая                      \\ \hline
			Теории игр     & $O(k \cdot a \cdot (v + e + |S_i|))$ & Средняя          & Высокая                      \\ \hline
			Обучения       & $O_{\text{обучение}}$      & Высокая                    & Высокая                      \\ \hline
		\end{tabular}
	\end{table}
\end{landscape}


\subsubsection*{Анализ результатов}
Рассмотрим свойства методов с учётом требований задачи:

1. **Потенциальные поля**:  
Метод обладает высокой гибкостью и сравнительно низкой сложностью, что делает его подходящим для задач в реальном времени. Однако правдоподобность поведения ограничена, так как агенты могут демонстрировать «нереалистичные» траектории движения из-за наличия локальных минимумов.

2. **Метод ролей**:  
Метод обеспечивает высокую правдоподобность благодаря детальному распределению задач, но страдает от низкой гибкости, так как агенты строго следуют заданным ролям. Также сложность возрастает с числом препятствий.

3. **Рой**:  
Метод роя демонстрирует высокую гибкость и устойчивость к изменениям среды, что полезно для динамических задач. Однако его квадратичная сложность по числу агентов ограничивает масштабируемость для больших систем.

4. **Планирование на основе теории игр**:  
Метод предоставляет наиболее правдоподобное поведение агентов благодаря равновесным стратегиям. Однако сложность метода остаётся высокой, особенно для больших систем с большим числом стратегий.

\subsubsection*{Рекомендации}
Для решения задачи координации в режиме реального времени рекомендуется использовать метод потенциальных полей или рой, учитывая их низкую сложность и высокую гибкость. Для задач, требующих высокой правдоподобности поведения (например, симуляции сложных игровых сценариев), более подходящим будет метод ролей или планирование на основе теории игр.
%\chapter{Конструкторская часть}
В данном разделе приведены диаграммы классов и схемы алгоритмов, необходимых для реализации программы.

\section{Диаграммы классов}
На рисунке \ref{img:classdiagram} представлена диаграммы классов основного модуля кодогенерации.

\includeimage
{classdiagram}
{f}
{H}
{\textwidth}
{Диаграмма классов модуля кодогенерации}

Главным классом является \texttt{CodeGenerator}, который последовательно применяет к входному SQL-выражению лексический анализатор, синтаксический анализатор, класс посещения абстрактного синтаксического дерева для сбора информации об SQL-выражении и класс преобразователя данных о выражении в код на языке \texttt{C++}.

Для применения разработанного модуля кодогенерации вместе с утилитой \texttt{Cog} разработан класс \texttt{CogPrinter}, который перенаправляет сгенерированный текст в поток вывода \texttt{Cog}, позволяя разместить его внутри файла с исходным кодом при помощи комментариев формата \texttt{Cog}.

В реализации интерфейса командной строки достаточно воспользоваться основным классом \texttt{CodeGenerator}, поскольку результаты его работы могут быть сразу перенаправлены в стандартный поток вывода для показа пользователю.

\section{Описание входных данных}
Разработанная библиотека должна поддерживать два формата работы -- вызов кодогенерации в виде функции программного модуля и приложение с интерфейсом командной строки.

В обоих случах входными данными для кодогенератора является текст SQL-выражения и имя итоговой сгенерированной сущности -- функции или класса.

Разработанное программное обеспечение должно поддерживать следующее подмножество языка SQL:
\begin{itemize}
	\item Инструкция CREATE TABLE со следующими поддерживаемыми типами столбцов:
	\begin{itemize}
		\item \texttt{BOOLEAN} -- логический тип размером 1 байт;
		\item \texttt{INT1} или \texttt{TINYINT} -- беззнаковое целое число размером 1 байт;
		\item \texttt{INT2} или \texttt{SMALLINT} -- беззнаковое целое число размером 2 байта;
		\item \texttt{INT3} или \texttt{MEDIUMINT} -- беззнаковое целое число размером 3 байта;
		\item \texttt{INT4} или \texttt{INT} -- беззнаковое целое число размером 4 байта;
		\item \texttt{INT8} или \texttt{BIGINT} -- беззнаковое целое число размером 8 байт.
	\end{itemize}
	\item Инструкция SELECT выборки данных из одной таблицы с условием WHERE;
	\item Инструкция INSERT INTO вставки данных в таблицу, поддерживающая вставку нескольких строк.
\end{itemize}

Все выражения, поступающие на вход кодогенераторы могут также содержать связываемые параметров, синтаксически задаваемые как \texttt{:foo}, где \texttt{foo} -- любой буквенный идентификатор \cite{sqlite_binding}.

Применение связываемых параметров позволяет параметризоваь создаваемые на основе SQL-выражений программные сущности.
При обращении к сущности в коде на C++ связываемые параметры будет необходимо задать в виде аргументов функции.

Поддержка работы только с одной таблицей обусловлена тем, что в рамках задач, обычно решаемых при помощи вычислительного комплекса <<Тераграф>>, зачастую не возникает необходимость смешивать структуры, так как в рамках одной структуры можно хранить различные данные в виде графа \cite{graph_practicum}.

\section{Способы взаимодействия с библиотекой}
В соответствии с диаграммой вариантов использования, представленной на рисунке \ref{img:usecase} необходимо спроектировать два вида интерфейсов для запуска кодогенерации.

Интерфейс командной строки должен предоставлять возможность кодогенерации на основе вводимых пользователем SQL-выражений.
Пользователь может вводить SQL-выражения построчно, в таком случае признаком окончания ввода является пустая строка.

После ввода SQL-выражения пользователю предлагается ввести имя для генерируемого объекта, либо оставить имя по-умолчанию, которое будет создано автоматически.
Необходимо предусмотреть возможность завершения работы с программой в соответствии с общепринятыми стандартами работы с интерфейсами командной строки.

Для того, чтобы утилиты кодогенерации могли вызывать исполняемый код библиотеки, должны быть разработанны классы-обёртки для поддержки конкретных утилит и описаны способы вызова процесса кодогенерации из них.

\section*{Вывод по конструкторской части}
Была разработана диаграмма классов и структура программного модуля библиотеки кодогенерации, учитывающие варианты использования программы.
Сформулировано описание природы входных данных системы и способов взаимодействия с ней в соответствии с диаграммами вариантов использования.
%\chapter{Технологическая часть}
В данном разделе обоснован выбор средств разработки ПО.

\section{Средства разработки}
Для разработки ПО выбран язык \texttt{Python} \cite{python}.
Данный язык программирования поддерживает объектно-ориентированнный подход и является интерпретируемым, что позволяет применять его в качестве скриптового встраиваемого языка.

В качестве инструмента для кодогенерации выбрана утилита \texttt{Cog} \cite{cog}, позволяющая производить запуск кода на \texttt{Python} в качестве этапа предобработки исходного кода на другом языке программирования при размещении его в комментариях специального формата.

Для произведения предварительного анализа текста SQL-выражений используется инструмент распознавания языков \texttt{ANTLR} \cite{antlr}, при помощи которого сгенерированы файлы исходного кода лексического и синтаксического анализаторов, а также базовый класс посетителя абстрактного синтаксического дерева на основе свободно распространяемых грамматик SQLite \cite{antlr_grammars}.

В качестве среды разработки выбрана PyCharm -- кросплатформенная IDE для \texttt{Python}, предоставляющая бесплатную лицензию студентам и хорошо знакомая мне, поскольку использовалась во время работы и учебы \cite{pycharm}.

Доступа к вычислительному комплексу Тераграф осуществляется с использованием облачной платформы Тераграф Cloud, обеспечивающей одновременный доступ многих пользователей к ядрам, входящим в состав микропроцессора Леонард Эйлер \cite{teragraph_cloud}.

Для подключения к платформе использовалась веб-версия IDE Visual Studio Code \cite{vscode}, запуск которой осуществляется при помощи подключения к серверу JupyterHub \cite{jupyter}.

\section{Формирование набора данных о SQL-выражении}
За формирование набора данных о SQL-выражении отвечает класс посетителя, точка входа которого приведена в листинге \ref{lst:visit_sql.py}.

В данном методе происходит обработка всех трех поддерживаемых типов SQL-выражений.

\includelisting
{visit_sql.py}
{Исходный код точки входа посетителя}

Функция сбора данных о выражении создания таблицы сохраняет название таблицы, информацию о первичном ключе и список столбцов таблицы с метаданными об их типе и размере в байтах.
Исходный код функции представлен в листинге \ref{lst:visit_create.py}.

\includelistingpretty
{visit_create.py}
{Python}
{Исходный код функции посещения выражения \texttt{CREATE TABLE}}

Для получения информации о типе данных столбца используется вспомогательный класс \texttt{TypeConverter}, преобразующий имя типа в данные о нём, что показано в листинге \ref{lst:type_converter.py}.

При этом обеспечивается поддержка нескольких видов задания числовых типов данных, напрямую по их размерам в байтах, либо по более удобным для человека именам.

\includelistingpretty
{type_converter.py}
{Python}
{Исходный код класса преобразования типов}

Функция сбор аданных о выражении выборки данных сохраняет информацию о запрошенных столбцах таблицы и их псеводнимах, что показано в листинге \ref{lst:visit_select.py}.

Также в контекст запроса выборки данных сохраняются таблицы, использованные в запросе.

\includelistingpretty
{visit_select.py}
{Python}
{Исходный код функции посещения выражения \texttt{SELECT}}

При сборе данных о выражении вставки необходимо сохранить все вставляемые строки и их содержание -- листинг \ref{lst:visit_insert.py}.

\includelistingpretty
{visit_insert.py}
{Python}
{Исходный код функции посещения выражения \texttt{INSERT}}

Все встреченные значения, к которым относятча числовые литералы и связываемые параметры также сохраняются для дальнейшего использования при преобразовании выражений.

Отдельно сохраняются литералы, связываемые параметры и все значения в порядке следования в SQL-выражении, что позволяет подставлять их как при преобразовании \texttt{SELECT} (в качестве значений регистров), так и \texttt{INSERT} выражений (в качестве параметров функции вставки) -- листинг \ref{lst:exprs.py}.

\includelistingpretty
{exprs.py}
{Python}
{Исходный код функции посещения значений}

\section{Преобразование SQL-выражений}
В данном разделе описаны реализованные алгоритмы преоборазования набора данных об SQL-выражении в текст на языке \texttt{C++}, а также необходимых для их реализации вспомогательных функций.

\subsection{Создание таблицы}
Для объявления структуры хранения данных при работе с библитекой \texttt{leonhard x64 xrt} необходимо создать структуру языка \texttt{C++}, содержащую объявления пар ключ-значение и описание их состава.
При этом накладывается ограничение на размер ключа и значения -- ровно 64 бита \cite{teragraph}.

Однако существует метод расширения структур, позволяющий хранить записи большего размера путем разбиение их на группы и создания множества ключей, отличающихся старшими битами.

Для преобразования данных о столбцах в набор ключей и значений применяется последовательное заполнение групп, пока новый столбец не приведёт к превышению максимального размера структуры значения группы, в этом случае происходит создание новой группы.

Каждая полученная группа преобразуется в пару структур в процессе кодогенерации.
Также создаётся вызов макроопределения, создающего вспомогательные методы получения ключа и значения, соответствующих группе.
Исходный код функции выделения групп из списка столбцов таблицы приведен в листинге \ref{lst:groups.py}

\includelistingpretty
{groups.py}
{Python}
{Исходный код функции выделения групп}

\subsection{Выборка данных}
Результатом кодогенерации из SQL-выражения выборки данных является структура итератора языка \texttt{C++}, позволяющего производить обход всех результатов выборки в соответствии с общепринятыми стандартами программирования на этом языке.

В силу сложности самостоятельного формирования алгоритма обхода структуры данных на основе SQL-запроса, было решено генерировать алгоритм на языке \texttt{C++} на основе алгоритма, предлагаемого ядром базы данных SQLite для осуществления выборки в виде байткода \cite{sqlite_bytecode}.

В виртуальной машине SQLite для хранения промежуточных результатов используются пронумерованные регистры, которые решено симулировать в виде полей класса итератора.

Использование байткода SQLite в качестве основы для алгоритма сопряжено с трудностями, поскольку требует отказа от структурного подхода к программированию.
Это связано с тем, что многие алгоритмы, созданные SQLite требуют нетривиальные ветвления, которые трудно автоматически преобразовать к if.
Поэтому в сгенерированном коде итератора может применяться оператор \texttt{goto} языка \texttt{C++}.

Часть исходного кода функции сопоставления операций байткода SQLite и операций языка \texttt{C++} приведена в листинге \ref{lst:bytecode.py}. Полный листинг функции не приводится из-за её большого объёма.

\includelistingpretty
{bytecode.py}
{Python}
{Исходный код функции сопоставления байткода SQLite}


\subsection{Вставка данных}
Инструкция вставки языка SQL преобразуется в функцию вставки \texttt{lnh\_ins\_sync()} библиотеки \texttt{leonhard x64 xrt}.

При этом для каждой группы исходной таблицы происходит вставка одной строки в структуру, а разбиение вводимых данных на ключ и значение достигается при помощи проверки вхождения имени вставляемого столбца в первичный ключ.

Все связываемые параметры превращаются в аргументы функции в соответствии с порядком в заданном SQL-выражении, при этом учитывается только порядок первых вхождений связываемых параметров, а дубликаты игнорируются.

Исходный код метода формирования тела функции вставки данных в структуру хранения данных приведен в листинге \ref{lst:insert.py}.

\includelistingpretty
{insert.py}
{Python}
{Исходный код метода формироавния тела функции вставки}

\section{Интерфейс командной строки}
Разработанный интерфейс командной строки запрашивает у пользователя построчный ввод SQL-выражений и имени генерируемого объекта, после чего выводит текст на языке \texttt{C++}, соответствующий запросу.
Для завершения работы необходимо воспользоваться сочетанием клавиш \texttt{Ctrl + C}.
Пример работы с интерфейсом командной строки приведен на рисунке \ref{img:cli}.

\includeimage
{cli}
{f}
{H}
{0.8\textwidth}
{Работа с интерфейсом командной строки}

\section{Кодогенерация в файле исходного кода}
Для использования кодогенерации в файле исходного кода необходимо установить утилиту \texttt{Cog}.
Она требует размещения в файле исходного кода \texttt{C++} комментариев специального формата.

Каждый такой комментарий, содержащий исполняемый код на \texttt{Python} называется ячейкой.

В начале файла необходимо разместить ячейку, в которой происходит создание экземпляра генератора, поддерживающего подключение к указанному файлу базы данных SQLite -- листинг \ref{lst:cog_start.py}.

Это необходимо для создания сессии кодогенерации в контексте одного соединения с базой данных.

\includelistingpretty
{cog_start.py}
{Python}
{Ячейка инициализации кодогенератора}

Для генерации кода из SQL-выражения необходимо разместить ячейку с вызовом метода \texttt{translate} у экземпляра кодогенератора и передачей исходного SQL-выражения и желаемого имени объекта -- листинг \ref{lst:cog_create.py}.

\includelistingpretty
{cog_create.py}
{Python}
{Пример ячейки кодогенерации}

Разрыв соединения с экземпляром базы данных и его удаление происходит автоматически по окончанию работы с файлом исходного кода при помощи переопределенного деструктора класса \texttt{CogPrinter}.

\section*{Вывод по технологической части}
Была разработана программный модуль на языке \texttt{Python}, реализующий метод кодогенерации инструкций работы с графовой базой данных на основе SQL-выражений.
Разработан интерфейс командной строки и програмнный интерфейс для вызова механизма кодогенерации в качестве внешнего программного модуля из утилиты кодогенерации.
%\chapter{Исследовательская часть}
В данном разделе поставлена цель исследования, описано проведенное исследование и сделанные на его основе выводы.

\section{Цель исследования}
Цель исследования -- удостовериться в корректности генерируемых в ходе преобразования SQL-выражений исходных кодов и соответствия выполняемых ими операций семантике исходного SQL-выражения.

Типы запросов \texttt{CREATE}:
\begin{itemize}
	\item Создание таблицы с простым первичным ключом и размером строки до 64 бит;
	\item Создание таблицы с простым первичным ключом и размером строки более 64 бит;
	\item Создание таблицы с составным первичным ключом и размером строки до 64 бит;
	\item Создание таблицы с составным первичным ключом и размером строки более 64 бит.
\end{itemize}

Типы запросов \texttt{SELECT}:
\begin{itemize}
	\item Выборка столбца из таблицы;
	\item Выборка столбца из таблицы с условием, включащим другой столбец;
	\item Выборка столбца из таблицы с условием, включащим параметр;
	\item Выборка столбцов из таблицы с условием, включащим другой столбец;
	\item Выборка столбцов из таблицы с условием, включащим параметр;
	\item Выборка из таблицы с составным условием, включающим несколько столбцов;
	\item Выборка из таблицы с составным условием, включающим параметр;
	\item Выборка из таблицы с ограничением числа результатов.
\end{itemize}

Типы запросов \texttt{INSERT}:
\begin{itemize}
	\item Вставка одной строки с константными значениями;
	\item Вставка одной строки с параметрмаи;
	\item Вставка нескольких строк.
\end{itemize}

Проверка корректности сгенерированных исходных кодов проводится путем их компиляции и загрузки в качестве программного ядра \texttt{lnh64} и запуска обработчиков при помощи запроса из host-подсистемы.

\section{Результаты исследования}
В данном разделе представлены результаты исследования корректности генерируемых инструкций для трех поддерживаемых типов SQL-выражений.

\subsection{Создание таблицы}
Корректность генерации структур хранения данных проверим путем запуска кодогенерации и сравнения результата с ожидаемым -- таблица \ref{tbl:research_create}.

\begin{table}[H]
	\caption{Результат исследования создания таблицы}
	\label{tbl:research_create}
	\begin{tabular}{|p{6.5cm}|p{3.3cm}|p{3.3cm}|}
		\hline
		Запрос & Ожидаемый результат & Результат \\ \hline
		Создание таблицы с простым первичным ключом и размером строки 64 бита & Структура из одной группы & Структура из одной группы \\ \hline
		Создание таблицы с простым первичным ключом и размером строки 128 бит & Структура из двух групп & Структура из двух групп \\ \hline
		Создание таблицы с составным первичным ключом и размером строки 64 бита & Структура из одной группы & Структура из одной группы \\ \hline
		Создание таблицы с составным первичным ключом и размером строки 128 бит & Структура из двух групп & Структура из двух групп \\ \hline
	\end{tabular}
\end{table}

\subsection{Выборка данных}
Для тестирования сгенерированных итераторов языка \texttt{C++} на сооветствие запросу база данных была создана структура хранения данных при помощи кодогенерации на основе SQL-выражения, представленного в листинге \ref{lst:users.sql}. данными, указанными в таблице \ref{tbl:data}.

\includelistingpretty
{users.sql}
{SQL}
{SQL-выражение создания таблицы пользователей}

Заполнение базы данных было произведено тестовыми данными, приведенными в таблице \ref{tbl:data}.

\begin{table}[H]
	\caption{Тестовые данные}
	\label{tbl:data}
	\begin{tabular}{|c|c|c|c|}
		\hline
		Id & User & Role & Time \\ \hline
		0 & 0 & 0 & 0 \\ \hline
		1 & 1 & 1 & 1000 \\ \hline
		2 & 1 & 1 & 2000 \\ \hline
		3 & 1 & 1 & 3000 \\ \hline
		4 & 1 & 1 & 4000 \\ \hline
		5 & 2 & 4 & 999 \\ \hline
		6 & 6 & 5 & 6000 \\ \hline
		7 & 7 & 6 & 7000 \\ \hline
		8 & 8 & 7 & 8000 \\ \hline
		9 & 9 & 8 & 9000 \\ \hline
	\end{tabular}
\end{table}

Далее в листингах \ref{lst:select1.sql}-\ref{lst:select9.sql} приведены SQL-выражения и результаты их выполнения в виде числовых значений запрошенных столбцов для каждой строки, попавшей в выборку.
\pagebreak

\includelistingpretty
{select1.sql}
{SQL}
{Выборка столбца из таблицы}

\includelistingpretty
{select2.sql}
{SQL}
{Выборка столбца из таблицы с условием с другим столбцом}

\includelistingpretty
{select3.sql}
{SQL}
{Выборка столбца из таблицы с условием с параметром}
\pagebreak
\includelistingpretty
{select4.sql}
{SQL}
{Выборка столбцов из таблицы}

\includelistingpretty
{select5.sql}
{SQL}
{Выборка столбцов из таблицы с условием с другим столбцом}

\includelistingpretty
{select6.sql}
{SQL}
{Выборка столбцов из таблицы с условием с параметром}
\pagebreak
\includelistingpretty
{select7.sql}
{SQL}
{Выборка из таблицы с составным условием с разными столбцами}

\includelistingpretty
{select8.sql}
{SQL}
{Выборка из таблицы с составным условием с параметром}

\includelistingpretty
{select9.sql}
{SQL}
{Выборка из таблицы с ограничением числа результатов}

Все результаты выполнения запросов соответствуют ожидаемым, следовательно, данные типы запросов корректно обрабатываются кодогенератором.

\subsection{Вставка данных}
Вставка данных была протестирована при выполнении тестирования выборки данных, поскольку заполнение базы данных происходило путём вызова выражений вставки трёх видов.
Соответствующие SQL-выражения приведены в листинге \ref{lst:inserts.sql}.

\includelistingpretty
{inserts.sql}
{SQL}
{SQL-выражения вставки}

\section*{Вывод по исследовательской части}
В результате исследования показано, что разработанная реализация метода кодогенерации успешно создает инструкции доступа к данным в вычислительном комплексе <<Тераграф>> из SQL-выражений и позволяет осуществлять работу с ним при помощи базовых инструкций создания таблиц, вставки и выборки данных.
\chapter*{ЗАКЛЮЧЕНИЕ}
\addcontentsline{toc}{chapter}{ЗАКЛЮЧЕНИЕ}
\iffalse
В ходе выполнения работы была достигнута поставленная цель -- проведено сравнение методов координации агентов применительно к задаче визуального контроля критических областей.

В процессе выполнения были решены все поставленные задачи:
\begin{enumerate}[leftmargin=1.6\parindent]
	\item Проведен анализ предметной области и описана рассматриваемая многоагентная система.
	\item Выделены характеристики для классификации и сравнения методов координации агентов в многоагентной системе.
	\item Формализованы математические описания рассматриваемых методов.
	\item Проведен сравнительный анализ методов по ключевым характеристикам.
\end{enumerate}

Для решения задачи координации агентов в описанной в работе задаче рекомендуется применение метода, основанного на потенциальных полях.
\fi

\phantomsection
\phantomsection
\addcontentsline{toc}{chapter}{СПИСОК ИСПОЛЬЗОВАННЫХ ИСТОЧНИКОВ}
\printbibliography[title=СПИСОК ИСПОЛЬЗОВАННЫХ ИСТОЧНИКОВ]

\begin{appendices}
	\chapter{}
	Презентация к выпускной квалификационной работе состоит из 20 слайдов.
\end{appendices}

\end{document}