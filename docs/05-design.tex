\chapter{Конструкторская часть}
В данном разделе приведены диаграммы классов и схемы алгоритмов, необходимых для реализации программы.

\section{Диаграммы классов}
На рисунке \ref{img:classdiagram} представлена диаграммы классов основного модуля кодогенерации.

\includeimage
{classdiagram}
{f}
{H}
{\textwidth}
{Диаграмма классов модуля кодогенерации}

Главным классом является \texttt{CodeGenerator}, который последовательно применяет к входному SQL-выражению лексический анализатор, синтаксический анализатор, класс посещения абстрактного синтаксического дерева для сбора информации об SQL-выражении и класс преобразователя данных о выражении в код на языке \texttt{C++}.

Для применения разработанного модуля кодогенерации вместе с утилитой \texttt{Cog} разработан класс \texttt{CogPrinter}, который перенаправляет сгенерированный текст в поток вывода \texttt{Cog}, позволяя разместить его внутри файла с исходным кодом при помощи комментариев формата \texttt{Cog}.

В реализации интерфейса командной строки достаточно воспользоваться основным классом \texttt{CodeGenerator}, поскольку результаты его работы могут быть сразу перенаправлены в стандартный поток вывода для показа пользователю.

\section{Описание входных данных}
Разработанная библиотека должна поддерживать два формата работы -- вызов кодогенерации в виде функции программного модуля и приложение с интерфейсом командной строки.

В обоих случах входными данными для кодогенератора является текст SQL-выражения и имя итоговой сгенерированной сущности -- функции или класса.

Разработанное программное обеспечение должно поддерживать следующее подмножество языка SQL:
\begin{itemize}
	\item Инструкция CREATE TABLE со следующими поддерживаемыми типами столбцов:
	\begin{itemize}
		\item \texttt{BOOLEAN} -- логический тип размером 1 байт;
		\item \texttt{INT1} или \texttt{TINYINT} -- беззнаковое целое число размером 1 байт;
		\item \texttt{INT2} или \texttt{SMALLINT} -- беззнаковое целое число размером 2 байта;
		\item \texttt{INT3} или \texttt{MEDIUMINT} -- беззнаковое целое число размером 3 байта;
		\item \texttt{INT4} или \texttt{INT} -- беззнаковое целое число размером 4 байта;
		\item \texttt{INT8} или \texttt{BIGINT} -- беззнаковое целое число размером 8 байт.
	\end{itemize}
	\item Инструкция SELECT выборки данных из одной таблицы с условием WHERE;
	\item Инструкция INSERT INTO вставки данных в таблицу, поддерживающая вставку нескольких строк.
\end{itemize}

Все выражения, поступающие на вход кодогенераторы могут также содержать связываемые параметров, синтаксически задаваемые как \texttt{:foo}, где \texttt{foo} -- любой буквенный идентификатор \cite{sqlite_binding}.

Применение связываемых параметров позволяет параметризоваь создаваемые на основе SQL-выражений программные сущности.
При обращении к сущности в коде на C++ связываемые параметры будет необходимо задать в виде аргументов функции.

Поддержка работы только с одной таблицей обусловлена тем, что в рамках задач, обычно решаемых при помощи вычислительного комплекса <<Тераграф>>, зачастую не возникает необходимость смешивать структуры, так как в рамках одной структуры можно хранить различные данные в виде графа \cite{graph_practicum}.

\section{Способы взаимодействия с библиотекой}
В соответствии с диаграммой вариантов использования, представленной на рисунке \ref{img:usecase} необходимо спроектировать два вида интерфейсов для запуска кодогенерации.

Интерфейс командной строки должен предоставлять возможность кодогенерации на основе вводимых пользователем SQL-выражений.
Пользователь может вводить SQL-выражения построчно, в таком случае признаком окончания ввода является пустая строка.

После ввода SQL-выражения пользователю предлагается ввести имя для генерируемого объекта, либо оставить имя по-умолчанию, которое будет создано автоматически.
Необходимо предусмотреть возможность завершения работы с программой в соответствии с общепринятыми стандартами работы с интерфейсами командной строки.

Для того, чтобы утилиты кодогенерации могли вызывать исполняемый код библиотеки, должны быть разработанны классы-обёртки для поддержки конкретных утилит и описаны способы вызова процесса кодогенерации из них.

\section*{Вывод по конструкторской части}
Была разработана диаграмма классов и структура программного модуля библиотеки кодогенерации, учитывающие варианты использования программы.
Сформулировано описание природы входных данных системы и способов взаимодействия с ней в соответствии с диаграммами вариантов использования.