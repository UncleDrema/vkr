\iffalse
\section{Методы координации на основе потенциальных полей}

\subsection{Метод потенциальных полей}
Метод потенциальных полей основывается на создании виртуальных сил, которые управляют движением агентов. Притягивающие силы направляют агентов к целевым точкам, а отталкивающие силы помогают избегать препятствий. Этот метод подходит для задач с простой динамикой и отсутствием сложных препятствий.

\textbf{Преимущества:}
\begin{itemize}
	\item Простота реализации.
	\item Низкие вычислительные требования.
	\item Хорошая масштабируемость для задач с небольшой сложностью.
\end{itemize}

\textbf{Недостатки:}
\begin{itemize}
	\item Уязвимость к локальным минимумам.
	\item Ограниченные возможности для адаптации к динамическим изменениям.
\end{itemize}

\textbf{Особенности:} Метод эффективно работает в условиях статических сред с малым числом препятствий.

\textbf{Проблемы:} Локальные минимумы могут задерживать агентов, особенно в сложных средах.

\subsection{Метод совместного использования потенциальных полей}
Этот подход расширяет базовый метод, позволяя агентам делиться информацией о своих потенциальных полях. Это улучшает координацию и избегание препятствий за счет интеграции данных от соседних агентов.

\textbf{Преимущества:}
\begin{itemize}
	\item Увеличенная координация между агентами.
	\item Более эффективное избегание препятствий.
	\item Улучшенная устойчивость к локальным минимумам.
\end{itemize}

\textbf{Недостатки:}
\begin{itemize}
	\item Возрастает необходимость в обмене данными.
	\item Более сложная реализация по сравнению с базовым методом.
\end{itemize}

\textbf{Особенности:} Подходит для задач, где требуется координация группы агентов.

\textbf{Проблемы:} Высокая нагрузка на коммуникационную сеть может ограничивать масштабируемость.

\subsection{Эргодическое покрытие с использованием потенциальных полей}
Этот метод создает потенциальные поля на основе решений уравнения теплопроводности для равномерного распределения агентов в заданной области. Метод обеспечивает эргодическое покрытие, включающее избегание столкновений.

\textbf{Преимущества:}
\begin{itemize}
	\item Высокая плотность покрытия области.
	\item Интегрированное избегание столкновений.
	\item Устойчивость к динамическим изменениям.
\end{itemize}

\textbf{Недостатки:}
\begin{itemize}
	\item Высокие вычислительные требования.
	\item Сложность реализации.
\end{itemize}

\textbf{Особенности:} Подходит для задач с требованиями высокой плотности покрытия.

\textbf{Проблемы:} Может быть сложным для использования в реальном времени из-за вычислительной нагрузки.

\subsection{Энтропийное развертывание с потенциальными полями}
Этот подход сочетает энтропийное распределение агентов и потенциальные поля для равномерного покрытия. Агентам назначаются области ответственности с учетом плотности покрытия.

\textbf{Преимущества:}
\begin{itemize}
	\item Низкая вычислительная сложность.
	\item Хорошая масштабируемость.
	\item Устойчивость к локальным минимумам.
\end{itemize}

\textbf{Недостатки:}
\begin{itemize}
	\item Менее эффективно при высокой динамике среды.
	\item Ограниченная точность распределения в сложных условиях.
\end{itemize}

\textbf{Особенности:} Подходит для децентрализованных систем с умеренными требованиями.

\textbf{Проблемы:} Требуется доработка для применения в высокодинамичных средах.

\subsection{Многослойные потенциальные поля}
Метод включает использование нескольких информационных слоев для создания локальных и глобальных сил, что обеспечивает гибкую навигацию и координацию.

\textbf{Преимущества:}
\begin{itemize}
	\item Гибкость и адаптивность.
	\item Устойчивость к локальным минимумам.
	\item Хорошая масштабируемость.
\end{itemize}

\textbf{Недостатки:}
\begin{itemize}
	\item Высокая сложность реализации.
	\item Требует значительных вычислительных ресурсов.
\end{itemize}

\textbf{Особенности:} Подходит для сложных динамических сред.

\textbf{Проблемы:} Может быть неэффективным для задач с ограниченными ресурсами.

\section{Сводная таблица}

\begin{table}[h!]
	\centering
	\begin{tabular}{|l|c|c|c|c|c|c|}
		\hline
		\textbf{Метод} & \textbf{Тип управления} & \textbf{Гибкость} & \textbf{Обмен данными} & \textbf{Устойчивость к минимумам} & \textbf{Масштабируемость} & \textbf{Сложность} \\ \hline
		Потенциальные поля & Децентрализованный & Средняя & Минус & Минус & Средняя & Низкая \\ \hline
		Совместное использование & Децентрализованный & Средняя & Плюс & Плюс & Средняя & Низкая \\ \hline
		Эргодическое покрытие & Централизованный & Высокая & Плюс & Плюс & Высокая & Высокая \\ \hline
		Энтропийное развертывание & Децентрализованный & Высокая & Минус & Плюс & Высокая & Средняя \\ \hline
		Многослойные потенциальные поля & Децентрализованный & Высокая & Минус & Плюс & Средняя & Высокая \\ \hline
	\end{tabular}
	\caption{Сравнение методов координации на основе потенциальных полей.}
\end{table}

\section{Вывод}
Для достижения оптимального результата в задачах визуального контроля критических областей и реагирования на угрозы рекомендуется комбинировать базовый метод \textbf{потенциальных полей} с подходами, обеспечивающими:  
\begin{itemize}
	\item устойчивость к локальным минимумам (например, многослойные потенциальные поля);
	\item высокую плотность покрытия (например, эргодическое покрытие);
	\item снижение вычислительных затрат (например, энтропийное развертывание).
\end{itemize}
Это позволит создать адаптивную, масштабируемую и эффективную систему координации агентов.
\fi